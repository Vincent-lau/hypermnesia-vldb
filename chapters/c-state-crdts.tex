%TC:ignore
This chapter complements~\cref{subsec:bg state-based crdt} with more formal details
of state-based \acrshortpl{crdt}, along
with examples. 

\section{Properties}

A state-based \acrshort{crdt} communicates by propagating the entire state of 
the data type with other parties, and perform the merge operation with a merge
(sometimes called join) operator to converge towards the least upper bound of the 
two states. They are sometimes also referred to as \acrfullpl{cvrdt}. I start by
defining some of the mathematical properties that state-based \acrshortpl{crdt},
leading towards its convergence property (\cref{prop:crdt convergence}), which
are essential in eventual consistency.

\begin{definition}[\acrfull{lub}~\cite{shapiro2011CRDT}]

We define the \acrshort{lub} of two states \(x\) and \(y\), partially
ordered by \(\sqsubseteq_v\) to be \(m=x \sqcup_v y\), and there is no \(m'\) such that
\(x\sqsubseteq_v m' \land y\sqsubseteq_v m'\). The operator defined above \(\sqcup_v\)
is also called the join operator.

\end{definition}

We observe that the join operator is idempotent 
(\(x\sqcup y = x\sqcup y \sqcup y = x\sqcup x \sqcup y\)),
commutative (\(x\sqcup y = y\sqcup x\)) and associative.
We then define the join semilattice to be:

\begin{definition}[Join Semilattice~\cite{davey_priestley_2002}]
  An ordered set \((S, \sqsubseteq_v)\), equipped with a join operator \(\sqcup_v\) 
  is a join semilattice if and only if \(\forall x, y\in S, \exists m = x\sqcup_v y\)
\end{definition}

Now a state-based \acrshort{crdt} consists of a triple \((S, M, Q)\) where \(S\)
is a join semilattice, \(Q\) is a set of \emph{query functions} and \(M\) is a set of
\emph{mutators}, where each \(m\) takes in one state \(X\in S\) and produce a new state
\(X'=m(X)\). Moreover, mutators are defined to be \emph{inflations}, i.e., for
any \(m\) and \(X\), we have \(X\sqsubseteq m(X)\).

This definition gives us some nice properties of state-based \acrshortpl{crdt}.
In particular, if we define our merge operator to be the join operator over
the join semilattice, then our merge process is converging towards the \acrshort{lub} 
of the most recent values. 
As an example, we consider the ordered set of natural numbers as \((\mathbb N, <)\),
ordered by the less-than relation. Then this is indeed a join semilattice
since the \(\max\) operator acts as the \acrshort{lub} of every two natural numbers
\(x\) and \(y\), i.e.\ \(x\sqcup y = \max(x,y)\). This property of the join 
semilattice gives the following guarantee:

\begin{proposition}[\cite{shapiro2011CRDT}] \label{prop:crdt convergence}
  Any two object replicas of a state-based \acrshortpl{crdt} eventually converge, 
  assuming the
  system transmits payload infinitely often between pairs of replicas over
  eventually-reliable point-to-point channels.
\end{proposition}

Note that we still require delivery of states to guarantee convergence,
i.e.\ we need every message to eventually reach every replica of the cluster.
However, this is the only requirement, since the \verb|merge| operator is idempotent and
commutative, we can tolerate arbitrary message reordering and repeated delivery
of messages.

The downside of the state-based \acrshort{crdt} is that it requires the dissemination
of the entire state, which makes it not suitable for collections of large states,
such as a big set of elements. \citet{almeida2018DeltaCRDT} proposes a variant
of state-based \acrshortpl{crdt} to address this issue, which we discuss below.

\section{Delta-state CRDTs} \label{subsec:delta state-based}

\acrfullpl{dcrdt}~\cite{almeida2018DeltaCRDT} is a new kind of state-based 
\acrshortpl{crdt} that disseminates only the changing part of the operation as 
a \(\delta\)-state, hence reducing the communication cost.

\begin{definition}[Delta-mutator~\cite{almeida2018DeltaCRDT}]
  A delta-mutator \(m^\delta\) is a function, corresponding to an update operation, 
  which takes a state \(X\) in a join semilattice \(S\) as its parameter and returns
  a delta-mutation \(m^\delta(X) \in S\).
\end{definition}

\begin{definition}[Delta-group~\cite{almeida2018DeltaCRDT}]
  A delta-group is inductively defined as either a delta-mutation or a join of
  several delta-groups. 
\end{definition}

\begin{definition}[\(\delta\)-\acrshort{crdt}~\cite{almeida2018DeltaCRDT}]
  A \(\delta\)-\acrshort{crdt} consists of a triple \((S, M^\delta, Q)\), where
  \(S\) and \(Q\) are defined as before, and \(M^\delta\) is a set of delta-mutators.
  The state transition is defined to be one of the below:

  Joining with a delta-state:
  \[
    X' = X\sqcup m^\delta(X) 
  \]

  Joining with a delta-group \(D\):
  \[
    X' = X\sqcup D 
  \]
\end{definition}

This definition decouples the state transition from applying the mutation, since
now we first mutate to get a delta-state, then we apply the join to change our
state. The results of a \(\delta\)-mutator looks almost like producing an operation
in op-based \acrshortpl{crdt}, apart from the requirement that the result must
also be a state in the join semilattice.

This definition is sufficient if the \acrshort{crdt} that we want to build does
not require causal consistency, such as a counter. But more is needed if we do
care about causal order, like (the addition and deletion in a) set. We therefore 
make some additional constructions:

A \emph{causal context} (sometimes also called \emph{causal history}) is a collection 
of events/dots defined as follows~\cite{almeida2018DeltaCRDT}:

\begin{align*}
  \mathrm{CausalContext} &= \mathcal P(\mathbb I\times \mathbb N) \\
  \max_i(c) &= \max(\{n\ |\ (i, n)\in c\} \cup \{0\}) \\
  \mathrm{next}_i(c) &= (i, \max_i(c) + 1)
\end{align*}

\noindent where \(\mathbb I\) is the set of replica identifiers, \(\mathbb N\)
is the natural numbers. A causal context allows us to tag each event in our system
with a unique identifier (or a \emph{dot}) \((i, n)\), signalling that the event 
happens at replica \(i\) at point \(n\). 

A \emph{dot store} acts as a container for data-type specific information. It can be
queried with the \verb|dots| function which takes in a dot store and returns the 
set of dots in the store. There are three types of dot store, defined as~\cite{almeida2018DeltaCRDT}

\begin{align*}
  \mathrm{DotSet}:\mathrm{DotStore} &= \mathcal P(\mathbb I\times \mathbb N) \\
  \mathrm{DotFun}\langle V:\mathrm{Lattice}\rangle: \mathrm{DotStore} &= \mathcal P(\mathbb I\times \mathbb N) \hookrightarrow V \\ 
  \mathrm{DotMap}\langle K, V:\mathrm{DotStore}\rangle: \mathrm{DotStore} &= K\hookrightarrow V
\end{align*}

And we can now combine the causal context and dot store to construct states for
\acrshortpl{dcrdt}~\cite{almeida2018DeltaCRDT}:

\begin{listing}[htp]
  \centering
\begin{align*}
  \mathrm{Causal}\langle T:\mathrm{DotStore}\rangle &= T\times \mathrm{CausalContext} \\
  \mathbf{when} &\qquad T:\mathrm{DotSet} \\
  (s,c)\sqcup (s',c') &= ((s\cap s')\cup (s\setminus c')\cup(s'\setminus c),c\cup c') \\
  \mathbf{when} &\qquad T:\mathrm{DotMap}\langle \_, \_\rangle \\
  (m, c) \sqcup (m', c') &= (\{k\mapsto v(k)\ |\ k\in \mathrm{dom}\ m \cup \mathrm{dom}\ m' \land v(k) \neq \bot\}, c \cup c') \\
  \mathbf{where}&\ v(k) = \mathrm{fst}((m(k), c) \sqcup (m'(k), c))
\end{align*}
  \caption{The join semilattice for \acrshortpl{dcrdt}, adapted from~\cite{almeida2018DeltaCRDT}.}
  \label{lst:delta crdt lattice}
\end{listing}


\section{Example: Delta-State set CRDT}

With the causal context and dot store above, we can now define our delta-state
add-wins set in~\cref{lst:delta state set}. A \acrshort{dcrdt} add-wins set
consists of a pair of a dot map and a causal context.
For such a set, when we add an element, the delta mutator \(add^\delta\) generates a
singleton map from the element to a dot, and the causal context is only concerned
with the new tag and all the previous tags associated with the added element. When we
remove an element, the dot map is replaced with an empty map and the causal context
with all the previous tags associated with the element, this will cause the
element \(e\) to be removed during the next join (because it is in the causal context
but not in the dot store, according to the join rule in~\cref{lst:delta crdt lattice}). 
Look-up is just examining the
domain of the dot map, i.e.\ everything that is in the dot map \(m\) is considered to be
in the set.

\begin{listing}[htp]
  \begin{align*}
    \mathrm{AWSet}\langle E\rangle &= \mathrm{Causal}\langle \mathrm{DotMap}\langle E, \mathrm{DotSet} \rangle\rangle \\
    \mathrm{add}_i^\delta(e, (m, c)) &= (\{e\mapsto d\}, d\cup m(e))\quad \mathbf{where}\ d=\{\mathrm{next}_i(c)\} \\
    \mathrm{remove}_i^\delta(e, (m, c)) &= (\{\}, m(e)) \\
    \mathrm{clear}_i^\delta((m, c)) &= (\{\}, \mathrm{dots}(m)) \\
    \mathrm{elements((m, c))} &= \mathrm{dom}\ m
  \end{align*}
 \caption{\acrshort{dcrdt} set specification, adapted from~\cite{almeida2018DeltaCRDT}.} 
 \label{lst:delta state set}
\end{listing}

%TC:endignore