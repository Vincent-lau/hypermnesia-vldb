% This chapter may be called something else\ldots but in general the
% idea is that you have one (or a few) ``meat'' chapters which describe
% the work you did in technical detail.

\section{System implementation} \label{sec:impl}


\Cref{fig:hypermnesia arch} shows an overview of Hypermnesia's architecture. 
A client sends a request to one of the database nodes connected in the cluster, 
and this request is then processed by the \verb|mnesia_ec| 
module (\cref{subsec:bg mnesia arch,subsec:design api}), responsible
for calling the underlying pure op-based Set \acrshort{crdt} implementation.
In Hypermnesia, two op-based Set \acrshortpl{crdt}
are implemented: add-wins set (\verb|mnesia_pawset|) and remove-wins set (\verb|mnesia_prwset|)
(\cref{sec:impl pawset}), which are called by \verb|mnesia_ec| based on the appropriate
table type (\cref{lst:mnesia ec create table}). 
\verb|mnesia_ec| also calls the \verb|mnesia_causal| module that handles the causal 
broadcast (\cref{sec:impl cbcast}). Each of these modules is explained in more
detail in the following sections.


\begin{figure*}[htp]
  \begin{tikzpicture}[
    replica/.style={rectangle,rounded corners, draw=black, inner sep=5pt,
    minimum height=15, minimum width=20, align=center,fill=ACMOrange},
    client/.style={rectangle,rounded corners,inner sep=5pt, minimum size=15, align=center,
    draw=black,fill=ACMBlue!50},
    client2/.style={rectangle,rounded corners, draw=black, inner sep=5pt,
    minimum height=15, minimum width=60, align=center,fill=ACMBlue},
    replica2/.style={rectangle,rounded corners, draw=black, inner sep=5pt,
    minimum height=15, minimum width=60, align=center, fill=ACMOrange},
    net/.style={rectangle,rounded corners, draw=black, inner sep=5pt,
    minimum height=15, minimum width=150, align=center,fill=ACMGreen},
    node distance = 2cm,
    skip loop/.style={to path={-- ++(0,-1) -| (\tikztotarget)}},
    mod/.style={text width=1cm}
    ]

    \node[replica] (n1) {N1};
    \node[replica, below left=of n1] (n2) {N2};
    \node[replica, below right=of n1]  (n3) {N3};
    \node[replica, below right=2cm and 0cm of n2]  (n4) {N4};
    \node[replica, below left=2cm and 0cm of n3]  (n5) {N5};

    \path[bidir] (n1) edge (n2);
    \path[bidir] (n1) edge (n3);
    \path[bidir] (n1) edge (n4);
    \path[bidir] (n1) edge (n5);
    \path[bidir] (n2) edge (n3);
    \path[bidir] (n2) edge (n4);
    \path[bidir] (n2) edge (n5);
    \path[bidir] (n3) edge (n4);
    \path[bidir] (n3) edge (n5);
    \path[bidir] (n4) edge (n5);

    \node[client,left =of n1] (c1) {Client 1}
      edge[post,dashed] node[auto, swap, sloped] {request} (n1);
    \node[above=of n1] (cluster) {Mnesia cluster};
      
    \node[right=3cm of cluster] (node a) {Node 1};
    \node[client2, below=1cm of node a] (app a) {Client};
    \node[replica2, below=of app a] (mnesia a) {Mnesia}
      edge[pre] node[auto] {write} (app a)
      edge[post, in=320,out=280,loop] (mnesia a);
    \node[net, below right=2.5cm and 0.01cm of mnesia a, xshift=-15mm] (net) {Network}
      ($(net.north west) + (0.45, 0) $) edge[pre] node[auto,near start,swap] (bcast1) 
      {broadcast} (mnesia a);
    \node[above=0.3cm of bcast1] {apply};

    \node[right=2.5cm of node a] (node b) {Node 3};
    \node[client2, right=of app a] (app b) {Client};
    \node[replica2, right=of mnesia a] (mnesia b) {Mnesia}
      edge[in=220,out=260,loop, post] node[auto, near end] {} (mnesia b)
      edge[post, dashed] node[auto] (read) {read} ($(app b.south) - (0.06, 0) $)
      ($(net.north east) - (0.7, 0) $) edge[post] node[auto,near start] (bcast2) 
      {broadcast} (mnesia b);
      \node[above=0.3cm of bcast2] {apply};

    \path[dotted, thick] (n1.north east) edge (node a);
    \path[dotted,thick] (n3.south) edge (net.south west);

    \begin{scope}[node distance=1cm]
      \node[right=of node b] (mod) {Modules};
      \node[right=1.4cm of read,mod] (ec) {\texttt{mnesia\_ec} (\cref{subsec:bg mnesia arch,subsec:design api})};
      \node[right=0.3cm of mnesia b,mod,yshift=-0.4cm] (pawset) 
        {\texttt{mnesia\_pawset} \texttt{mnesia\_prwset} (\cref{sec:impl pawset})};
      \node[right=1.4 cm of bcast2,mod] (causal) {\texttt{mnesia\_causal} (\cref{sec:impl cbcast})};
    \end{scope}

  \end{tikzpicture}
  \caption{Hypermnesia architecture overview.}
  \label{fig:hypermnesia arch}
\end{figure*}


\subsection{Causal Broadcast} \label{sec:impl cbcast}

Causal broadcast is a mature algorithm with standard implementation 
strategies~\cite{schmuck1988broadcast,birman1991causal}. 
It is often treated as a middleware between the
network and the application. It buffers messages until they are causally ready
to be delivered to the application. The \mintinline{erlang}{gen_server} behaviour
is a suitable abstraction, as it provides an interface from which
developers can write custom implementations of request handlers to obtain a
generic server (similar to interface in object-oriented 
languages)~\cite{ericssonab2023otpdesign}.

More concretely, we define a collection of public functions for the causal broadcast
server, which can be called by \mintinline{erlang}{mnesia_ec} while sending
and receiving messages. Two most basic ones are \mintinline{erlang}{send_msg/1}
and \mintinline{erlang}{rcv_msg/1}, as shown in~\cref{lst:causal broadcast server}.
The \mintinline{erlang}{send_msg/1} function returns the current timestamp
as a vector clock to be attached to the message.  The \mintinline{erlang}{rcv_msg/1}
function takes a received message and returns a list of messages ready to be
delivered to Mnesia for further processing, such as database writing.
The \mintinline{erlang}{mnesia_causal} server also keeps track of a list of buffered messages.
Each time a message is received, it is added to the buffer, and the buffer is
searched for deliverable messages.


\begin{listing}[htp]
  \centering
  \begin{minted}[frame=lines]{erlang}
    -record(state,
        {send_seq :: integer(),
         delivered :: vclock(),
         buffer :: [msg()] }).

    -spec send_msg() -> vclock().
    -spec rcv_msg(msg()) -> [msg()].
  \end{minted} 
  \caption{\texttt{mnesia\_causal} server for causal broadcast.}
  \label{lst:causal broadcast server}
\end{listing}


\subsection{Tagged causal stable broadcast} \label{subsec:impl tcsb}

The algorithm discussed above is a standard causal broadcast 
algorithm~\cite{schmuck1988broadcast,kleppmann2022dist-notes,birman1991causal}.
It does not expose any ordering or timestamp information when delivering a message.
A Set \acrshort{crdt} often relies on unique identifiers to
ensure commutativity of causally concurrent operations. \citet{baquero2017PureOp}
observe that the timestamp information from the causal broadcast layer can act
as a unique identifier for the \acrshort{crdt}, and thus
propose exposing the ordering information from the causal broadcast layer to
the application, i.e.\ the receive function would return a list of messages
along with their timestamps (\cref{lst:tcsb}).

\begin{listing}[htp]
  \centering
  \begin{minted}[frame=lines]{erlang}
    -spec send_msg() -> timestamp().
    -spec rcv_msg(msg()) -> [{msg(), timestamp()}].
  \end{minted} 
  \caption{\texttt{mnesia\_causal} server interface for \acrlong{tcsb}.}
  \label{lst:tcsb}
\end{listing}

The extended \acrfull{tcsb} protocol works like this: when a message is ready 
to be sent, the function \verb|send_msg/0| is called to obtain the timestamp which
is attached to the message. 
And when a message is received, the \verb|rcv_msg/1| is called to buffer the message
and find all messages ready to be delivered. A message is ready when
the following two conditions hold~\cite{schmuck1988broadcast,
kleppmann2022dist-notes,birman1991causal}:

\begin{enumerate}
  \item The sender entry in the receiver's delivered map, which keeps track of how many
  messages are delivered for each sender, is exactly one less than the entry
  in the received message's timestamp;
  \item The other entries in the received message's timestamp are all less than
  or equal to the corresponding entries in the receiver's delivered map.
\end{enumerate}

The first condition ensures that all causally preceding messages from the sender
have been delivered, and the second condition prevents us from delivering 
messages that causally depend on messages from other nodes. Note that the sender 
of the message can always deliver messages from itself immediately, a desirable 
property for
availability during partitions, and causal consistency is the strongest consistency
model that provides such always-available property~\cite{mahajan2012cac}.

\subsection{Pure operation-based Set CRDT} \label{sec:impl pawset}

We have discussed the idea of a pure op-based set (\cref{subsec:bg op-based crdts}). 
This section builds on top of the basic
ideas of op-based Set \acrshort{crdt} and highlights 
a few optimisations exploiting the causal broadcast 
layer (\crefrange{subsec:impl causal redundancy}{subsec:impl pawset algorithm}).
More subtle implementation challenges in interactions
between Mnesia and the pure op-based set \acrshort{crdt} are also 
discussed (\cref{subsec:impl practical}).

\subsection{Causal Redundancy} \label{subsec:impl causal redundancy}

\emph{Causal redundancy} is proposed by~\citet{baquero2017PureOp} for elements in
the \acrshort{polog} that can be removed without affecting the output of
queries. We define the
redundancy relation in~\cref{lst:aw set causal redundancy}. The first rule
says that a clear or remove operation is redundant in all cases. The second rule
makes causally older additions of the same elements redundant. The third rule says every
addition of an element present in the \acrshort{polog} is in the set. Redundant
elements can be safely removed from the \acrshort{polog}, which helps reduce the
storage cost.

\begin{listing}[htp]
  \centering
  \begin{align*}
    (t,o)\ R\ s &\iff o[0] = \mathrm{clear} \lor \mathrm{rmv}\\
    (t,o)\ R\_\ (t',o') &\iff t<t' \land (o'[0]=\mathrm{clear}\lor o[1]=o'[1]) \\
    \mathrm{eval}_i(\mathrm{elems},s) &= \{v\ \mid\ (\_,[\mathrm{add},v])\in s\}
  \end{align*} 
  \caption{The redundancy relation \(R\) for an \acrshort{awset}. \(o[0]\) is the
  operation, while \(o[1]\) is the key. Taken from~\cite{baquero2017PureOp}.} 
  \label{lst:aw set causal redundancy}
\end{listing}


\subsection{Causal stability}  \label{subsec:impl causal stability}

The number of timestamps associated with each element grows as elements are added
to the set. Moreover, each of them is proportional to the number of nodes in the 
cluster. \citet{baquero2014PureOp} suggest removing the timestamps that are
\emph{causally stable}~\cite{shapiro2011CRDT,baquero2017PureOp}:

\begin{definition}[Causal Stability~\cite{baquero2017PureOp,baquero2014PureOp}]
  A timestamp \(\tau\) and a corresponding message, is causally stable at node 
  \(i\) when all messages subsequently delivered at \(i\) will have timestamp 
  \(t>\tau\).
  Or equivalently:

  \[
    \mathrm{tcstable}_i(\tau)\ \mathbf{if}\ \forall j\in I\setminus \{i\}. \exists
    t\in \mathrm{deliv}_i()\cdot\mathrm{src}(t) = j \land \tau < t.
  \]
\end{definition}

Once a timestamp is stable, it can be removed. This is safe since there will be no more
concurrent operations delivered in the future and the timestamp metadata is
used to ensure commutativity for concurrent operations and is therefore no longer
needed, reducing the storage cost of the \acrshort{polog}.

The actual implementation in Hypermnesia is done by asking the 
pure \acrshort{awset} periodically scanning elements and removing causally stable timestamps.

\subsection{Broadcast and CRDT algorithms} \label{subsec:impl pawset algorithm}

\Cref{algo:tcsb} shows the broadcast algorithm obtained by putting these optimisations
together (\crefrange{subsec:impl tcsb}{subsec:impl causal stability}). 
When broadcasting a message (\crefrange{line:broadcast start}{line:broadcast end}),
the appropriate timestamp and sender id are attached to the message, which is then
received and buffered (\crefrange{line:receive start}{line:receive buffer end}). The
receiver searches for messages that are causally ready to be delivered using
the condition in~\cref{subsec:impl tcsb} (\crefrange{line:receive search start}{line:receive search end}).
The causal broadcast server also provides a function to check for the stability of a 
timestamp (\crefrange{line:stable start}{line:stable end}).
This algorithm combines the standard causal broadcast
algorithm~\cite{birman1991causal,kleppmann2022dist-notes} and extra exposed 
APIs proposed by~\cite{baquero2014PureOp,baquero2017PureOp}, adapted to fit Mnesia's 
architecture (\cref{subsec:bg mnesia arch}).

\Cref{algo:pawset} shows the corresponding algorithm for the pure \acrshort{awset},
written according to the specification~\cite{baquero2017PureOp,baquero2014PureOp}.
The original specifications
are written as a mathematical specification, while we take
a programming perspective and present them in terms of set operations like
\texttt{add}, \texttt{delete}, etc. The \verb|remove_redundant| function is called
each time a deletion or insertion happens, and checks
for redundant elements using the \verb|redundant| 
function (\crefrange{line:awset red start}{line:awset red end}), based on the 
idea of causal stability (\cref{lst:aw set causal redundancy}).

\begin{algorithm}[htp]
  \DontPrintSemicolon
  \SetKwProg{Fn}{on}{:}{}  
  \SetKwData{Delivered}{delivered} \SetKwData{Buffer}{buffer} \SetKwData{SendSeq}{sendSeq}
  \SetKwData{Msg}{msg} \SetKwData{Sender}{sender} \SetKwData{Deps}{deps}
  \SetKwData{Deliverable}{deliverable} \SetKwData{Ts}{ts}
  \SetKwData{tsmap}{ts\_map}
  \SetKwData{true}{true} \SetKwData{false}{false}

  \SetKwFunction{Src}{src}

  \Fn(){init} { \nllabel{line:init start}
    \(\SendSeq \longleftarrow 0\) \;
    \(\Buffer \longleftarrow \emptyset\) \;
    \(\Delivered \longleftarrow \{(s, 0)\ \mid\  \forall s \in \mathrm{nodes}\}\) \;
    \tcc{a map from node to last the timestamp of the last delivered message}
    \(\tsmap \longleftarrow \{(s, \bot\ \mid\ \forall s\in \mathrm{nodes})\}\) \;
  }  \nllabel{line:init end}

  \Fn(){broadcast \Msg at replica \(i\)}{ \nllabel{line:broadcast start}
    \(\Delivered[i] \leftarrow \SendSeq + 1\) \;
    \SendSeq \(\leftarrow\) \SendSeq \(+ 1\) \;
    \Deps \(\leftarrow\) \Delivered \;
    broadcast (\Msg, \(i\), \Deps) to all nodes \;
  } \nllabel{line:broadcast end}

  \Fn(){receive (\Msg,\Sender,\Deps) at replica \(i\) } { \nllabel{line:receive start}
    \Buffer \(\leftarrow\) \Buffer \(\cup\) \((\Msg,\Sender,\Deps)\) \; \nllabel{line:receive buffer end}
    \Repeat{\(D = \emptyset\)}{ \nllabel{line:receive search start}
      \tcc{find all causally deliverable messages}
      \(D \leftarrow \{(\Msg,\Sender,\tau)\ |\ \exists (\Msg,\Sender,\tau)\in \Buffer.
      \Deps[\Sender] = \tau[\Sender] + 1 \land \forall s \in  \mathrm{dom}(\Deps)
      \setminus \{\Sender\}.\ \Deps[s] \geq \tau[s]\}\) \;
      \tcc{update delivered vector and timestamp map}
      \For{\((\Msg,\Sender,\tau) \in D\)}{
        \(\Delivered[\Sender] \leftarrow \Delivered[\Sender] + 1\) \;
        \(\tsmap[\Sender] \leftarrow \tau\) \;
      }
      \(\Buffer \leftarrow \Buffer \setminus D\) \;
      \(\Deliverable \leftarrow \Deliverable \cup D \) \;
    } \nllabel{line:receive search end}
    deliver \Deliverable to application \;
  } \nllabel{line:receive end}

  \Fn(){check stability of \(\tau\)} { \nllabel{line:stable start}
    \If{\(\tau \leq \min(\{\tsmap[s][\Src{\(\tau\)}]\ \mid\ s\in\mathrm{dom}(\tsmap)\})\)}{
      \KwRet{\true} \;
    } \Else{
      \KwRet{\false} \;
    }
  } \nllabel{line:stable end}


  \caption{\acrfull{tcsb} protocol algorithm, ideas taken
  from~\cite{kleppmann2022dist-notes,baquero2017PureOp,bauwens2021Reactivity}.}
  \label{algo:tcsb}
\end{algorithm}


\begin{algorithm}[htp]
  \DontPrintSemicolon
  \SetKwProg{Fn}{function}{:}{}  

  \SetKwFunction{Add}{add} \SetKwFunction{Delete}{delete} \SetKwFunction{Read}{read}
  \SetKwFunction{Stable}{stable}
  \SetKwFunction{Red}{redundant} 
  \SetKwFunction{RemoveRed}{remove\_redundant}
  \SetKwFunction{App}{append} \SetKwFunction{Rmv}{remove}
  \SetKwRepeat{Repeat}{periodically}{end}


  \SetKwData{Msg}{msg} \SetKwData{Sender}{sender}
  \SetKwData{polog}{POLog}
  \SetKwData{true}{true} \SetKwData{false}{false}
  \SetKwData{noop}{no-op} \SetKwData{add}{add} \SetKwData{delete}{delete}
  \SetKwData{clear}{clear}

  \(\polog \leftarrow []\) \;


  \Fn(){\Add{\(e, t\)}} {
    \RemoveRed{\(e, t, \add\)} \;
    \For{\((e',t',o')\in \polog\)}{
      \If{\Red{\((e,t,\add),(e',t',o')\)}}{
        \KwRet{} \;
      }
    }
    \(\polog \leftarrow \App{\polog, \((e,t,\add)\)} \) \;
  }

  \Fn(){\Delete(\(e, t\))} {
    \RemoveRed{\(e, t, \delete\)}\;

    \For{\((e',t',o')\in \polog\)}{
      \If{\Red{\((e,t,\delete),(e',t',o')\)}}{
        \KwRet{} \;
      }
    }
    \(\polog \leftarrow \App{\polog, \((e,t,\delete)\)} \) \;

  }
   
  \Fn(){\Read{\(k\)}} {
    \For{\((e,t,o) \in \polog\)}{
      \If{\(e.\mathrm{key} = k\)} {
        \KwRet{\(e\)} \;
      }
    }
    \KwRet{\(\mathrm{undefined}\)} \;
  }

  \Fn(){\RemoveRed(\(e,t,o\))} {
    \For{\((e',t',o') \in \polog\)}{\nllabel{line:awset remove red}
      \If{\Red{\((e',t',o'),(e,t,o)\)}}{
        \(\polog \leftarrow \Rmv{\(\polog, (e',t',o')\)}\) \;
      } 
    }
  }

  \Fn(){\Red((e,t,o), (e',t',o'))} { \nllabel{line:awset red start}
    \tcc{check whether \((e,t,o)\) is made redundant by \((e',t',o')\)}
     \If{\(o=\delete\)}{
        \KwRet{\true} \;
     } \ElseIf{\(e = e' \land t < t'\)}{\nllabel{line:awset red}
        \KwRet{\true} \;
     } \Else{
        \KwRet{\false} \;
     }
  } \nllabel{line:awset red end}

  
  \Repeat{}{
    \For{\((e,t,o) \in \polog\)}{
      \If{\Stable(\(t\))}{
        \(\polog \leftarrow \Rmv{\(\polog, (e,t,o)\)}\) \;
        \(\polog \leftarrow \App{\(\polog, (e',\bot,\noop)\)}\) \;
      }
    }
  }

  \caption{Pure \acrshort{awset} pseudocode, defined in terms of usual set operations.
  This pseudocode sacrifices efficiency for clarity. Ideas are taken 
  from~\cite{baquero2017PureOp,baquero2014PureOp,bauwens2021Reactivity}.}
  \label{algo:pawset}
\end{algorithm}

Apart from the \acrlong{awset}, a \acrlong{rwset} (\cref{fig:hypermnesia arch}) 
is also implemented, and the algorithm is mostly similar to that of an \acrlong{awset}.

\subsection{Practical concerns} \label{subsec:impl practical}

This section discusses several practical issues in implementing the algorithm.

\paragraph{Faster access to the \acrshort{polog}}

As discussed in~\cref{subsec:bg mnesia arch}, Mnesia uses \texttt{ets} and \texttt{dets}
as its storage system, which are implemented as hash tables and balanced binary trees.
There is no direct support for log-like structures such as lists. This
presents challenges as to how the \acrshort{polog} could be implemented. Typically
a Mnesia table is a set indexed by the \(n\)-th element in a tuple, where \(n\) 
is customisable. We keep this structure but use a bag instead of a set to allow
multiple elements with the same key but different (concurrent) timestamps. 
This representation gives us the advantage of accessing the element by key 
in \(\mathcal O(1)\) time, which is useful in speeding
up the \verb|remove_redundant| function that needs a liner scan of the entire set
for matching keys (\cref{line:awset remove red} in~\cref{algo:pawset}).
However, this representation also loses the partial order of the
\acrshort{polog}, making the stabilisation process (\cref{subsec:impl causal stability}) 
much more difficult. 


\paragraph{Payload conflicts}

\Cref{algo:pawset} deals with operations of a set in terms of addition, deletion,
etc. Although the Mnesia documentation claims the data structure to be a set,
it behaves more like a map, with keys and payload 
(recall from~\cref{subsec:bg mnesia data repr} we saw that each row in the student 
table has an id as the key, and other information like name as the payload). 
Therefore the Set \acrshort{crdt}
falls short when there are concurrent additions of the same key but different
payloads. This can be solved in general with a Map \acrshort{crdt}~\cite{shapiro2011CRDT}, 
which is a straightforward extension of a Set \acrshort{crdt}, but requires the
payload to be \acrshortpl{crdt} as well so that the conflict can be resolved
recursively for the payload.

This puts constraints on the data a user can put in a Mnesia table, and
is considered as breaking too much compatibility of the existing system and therefore
I take a simpler approach by resolving them based on the Erlang term 
order~\cite{ericssonab2023refmanual}, an ordering of data types in Erlang.
This could easily be extended with other resolution strategies as well.


\paragraph{More operations on Mnesia tables}

Apart from the basic addition and deletion of elements, Mnesia 
supports a range of other operations as well, including
matching based on a pattern specification, iterating over a table, finding all
the keys of a table, etc. As an example, the original \verb|select/2| function
in Mnesia takes a table and a pattern specification as its argument. This is
modified by appending wild cards for the timestamp and operation name (since they
are stored along with each tuple in the \acrshort{polog}) to the input pattern 
before matching. Other functions like \verb|all_keys/1| are implemented with
similar ideas.


\subsection{Fault tolerance} \label{sec:impl fault tolerance}

This section continues the discussion from~\cref{subsec:bg cap} on Mnesia's response
to network partitions and talks about how Hypermnesia addresses them. We continue
to use the term transient failure and communication failure to distinguish them.

\subsection{Communication failure} \label{subsec:impl communication failure}

\Cref{tab:mnesia comm failure ops} shows the sequence of
operations and the corresponding states of each replica, and
\cref{fig:mnesia comm failure} shows the graphical representation where three
Mnesia nodes are initially connected to each other and each has an element
\texttt{a} in them. Then a partition occurs between node A and node B, as well as
node A and node C. During this partition, an addition of 
element \texttt{c} occurs at A while addition of element \texttt{b} happens
at B and C. When the partition recovers, replicas are now in an inconsistent
state which will be reported to the developer for manual resolution.


\begin{table}[htp]
  \begin{tabular}{cccc}
    \toprule
    Operation & Node A & Node B & Node C \\
    \midrule
    A::add(a) & \{a\} & \{a\} & \{a\} \\
    \cmidrule{2-4}
    B::add(b) & \{a\} & \{a,b\} & \{a,b\} \\
    A::add(c) & \{a,c\} & \{a,b\} & \{a,b\} \\
    \cmidrule{2-4}
    \texttt{inconsistent\_database} & \{a,c\} & \{a,b\} & \{a,b\} \\
    \bottomrule
  \end{tabular} 
  \caption{Operations performed on each replica. Operations that happen during
  the network partition sit between two horizontal lines.}
  \label{tab:mnesia comm failure ops}
\end{table}

\begin{figure}[htp]
  \begin{subfigure}[t]{0.3\textwidth}
    \centering
    \begin{tikzpicture}[
      replica/.style={rectangle,rounded corners, draw=black, inner sep=5pt,
      minimum height=15, minimum width=20, align=center,fill=ACMOrange},
    ]
      \node[replica] (s1-m1) {A};
      \node[replica,below left=1cm and 0.5cm of s1-m1] (s1-m2) {B}
        edge[bidir] (s1-m1);
      \node[replica,below right=1cm and 0.5cm of s1-m1] (s1-m3) {C}
        edge[bidir] (s1-m1)
        edge[bidir] (s1-m2);
  
      \node[above=1mm of s1-m1] (s1-t1) {\{x\}};
      \node[below=1mm of s1-m2] (s1-t2) {\{x\}};
      \node[below=1mm of s1-m3] (s1-t3) {\{x\}};
    \end{tikzpicture} 
    \caption{Initially every replica has an \texttt{a} in its set.}
  \end{subfigure}
  
  \begin{subfigure}[t]{0.3\textwidth}
    \centering
    \begin{tikzpicture}[
      replica/.style={rectangle,rounded corners, draw=black, inner sep=5pt,
      minimum height=15, minimum width=20, align=center,fill=ACMOrange},
    ]
      \node[replica,right=4cm of s1-m1] (s2-m1) {A};
      \node[replica,below left=1cm and 0.5cm of s2-m1] (s2-m2) {B}
        edge[bidir,dashed] (s2-m1);
      \node[replica,below right=1cm and 0.5cm of s2-m1] (s2-m3) {C}
        edge[bidir, dashed] (s2-m1)
        edge[bidir] (s2-m2);

      \node[above=1mm of s2-m1] (s2-t1) {\{x,\textcolor{ACMGreen}{z}\}};
      \node[below=1mm of s2-m2] (s2-t2) {\{x,\textcolor{ACMGreen}{y}\}};
      \node[below=1mm of s2-m3] (s2-t3) {\{x,\textcolor{ACMGreen}{y}\}};
    \end{tikzpicture}
    \caption{A partition (dashed lines) making A temporarily unreachable. Meanwhile,
    \texttt{y} and element \texttt{z} are added to the set.}
  \end{subfigure}
  
  \begin{subfigure}[t]{0.3\textwidth}
    \centering
    \begin{tikzpicture}[
      replica/.style={rectangle,rounded corners, draw=black, inner sep=5pt,
      minimum height=15, minimum width=20, align=center,fill=ACMOrange},
    ]
      \node[replica] (s3-m1) {A};
      \node[replica,below left=1cm and 0.5cm of s3-m1] (s3-m2) {B}
        edge[bidir] (s3-m1);
      \node[replica,below right=1cm and 0.5cm of s3-m1] (s3-m3) {C}
        edge[bidir] (s3-m1)
        edge[bidir] (s3-m2);
  
      \node[above=1mm of s3-m1] (s3-t1) {\{x,z\}};
      \node[below=1mm of s3-m2] (s3-t2) {\{x,y\}};
      \node[below=1mm of s3-m3] (s3-t3) {\{x,y\}};

      \node[left=1mm of s3-m1] (s3-w1) {\color{ACMRed}{Error!}};

    \end{tikzpicture} 
    \caption{When the partition heals, Mnesia would log an error message indicating
    \texttt{inconsistent\_database} to the user for manual resolution.}
  \end{subfigure}
  \caption{Illustration of a communication failure in Mnesia.}
  \label{fig:mnesia comm failure}
\end{figure}

\begin{figure}[htp]
  \begin{subfigure}[t]{0.9\columnwidth}
    \centering
    \begin{tikzpicture}[
      replica/.style={rectangle,rounded corners, draw=black, inner sep=5pt,
      minimum height=15, minimum width=20, align=center,fill=ACMOrange},
    ]
      \node[replica] (s1-m1) {A};
      \node[replica,below left=1cm and 0.5cm of s1-m1] (s1-m2) {B}
        edge[bidir] (s1-m1);
      \node[replica,below right=1cm and 0.5cm of s1-m1] (s1-m3) {C}
        edge[bidir] (s1-m1)
        edge[bidir] (s1-m2);
  
      \node[above=1mm of s1-m1] (s1-t1) {\{x\}};
      \node[below=1mm of s1-m2] (s1-t2) {\{x\}};
      \node[below=1mm of s1-m3] (s1-t3) {\{x\}};
    \end{tikzpicture} 
    \caption{Initially every replica has an \texttt{a} in its set.}
  \end{subfigure}

  \begin{subfigure}[t]{0.9\columnwidth}
    \centering
    \begin{tikzpicture}[
      replica/.style={rectangle,rounded corners, draw=black, inner sep=5pt,
      minimum height=15, minimum width=20, align=center,fill=ACMOrange},
    ]
      \node[replica] (s2-m1) {A};
      \node[replica,below left=1cm and 0.5cm of s2-m1] (s2-m2) {B}
        edge[bidir,dashed] (s2-m1);
      \node[replica,below right=1cm and 0.5cm of s2-m1] (s2-m3) {C}
        edge[bidir, dashed] (s2-m1)
        edge[bidir] (s2-m2);

      \node[above=1mm of s2-m1] (s2-t1) {\{x,\textcolor{ACMGreen}{z}\}};
      \node[below=1mm of s2-m2] (s2-t2) {\{x,\textcolor{ACMGreen}{y}\}};
      \node[below=1mm of s2-m3] (s2-t3) {\{x,\textcolor{ACMGreen}{y}\}};

      \node[left=1mm of s2-m1] (s2-o1) {add(z)\texttt{->}B};
      \node[right=1mm of s2-m1] (s2-o1) {add(z)\texttt{->}C};
      \node[below=1mm of s2-t2] (s2-o1) {add(y)\texttt{->}A};

    \end{tikzpicture}
    \caption{Hypermnesia buffers messages during the partition.}
    \label{subfig:hypermnesia comm failure b}
  \end{subfigure}

  \begin{subfigure}[t]{0.9\columnwidth}
    \centering
    \begin{tikzpicture}[
      replica/.style={rectangle,rounded corners, draw=black, inner sep=5pt,
      minimum height=15, minimum width=20, align=center,fill=ACMOrange},
    ]
      \node[replica] (s3-m1) {A};
      \node[replica,below left=1cm and 0.5cm of s3-m1] (s3-m2) {B}
        edge[bidir] (s3-m1);
      \node[replica,below right=1cm and 0.5cm of s3-m1] (s3-m3) {C}
        edge[bidir] (s3-m1)
        edge[bidir] (s3-m2);
  
      \node[above=1mm of s3-m1] (s3-t1) {\{x,y,z\}};
      \node[below=1mm of s3-m2] (s3-t2) {\{x,y,z\}};
      \node[below=1mm of s3-m3] (s3-t3) {\{x,y,z\}};

    \end{tikzpicture} 
    \caption{When the communication failure recovers, buffered messages are delivered
    and conflicts are resolved.}
    \label{fig:hypermnesia comm failure}
  \end{subfigure}
  \caption{Hypermnesia buffers messages during communication failure and resolves
  conflicts afterwards.}
\end{figure}


Hypermnesia resolves this issue by buffering messages during the partition, as
shown in~\cref{fig:hypermnesia comm failure}. During the communication failure, 
operations performed on replicas A and B are buffered until the partition heals. 
By the property of op-based \acrshortpl{crdt} (\cref{subsec:bg op-based crdts}),
replicas will converge after the buffered messages are delivered.
In this simple case, it is sufficient to achieve consistency by a simple set 
union. In a more complex case
such as concurrent addition and deletion, the underlying op-based 
\acrshort{crdt} (\cref{sec:impl pawset}) will handle the conflict resolution 
logic to ensure convergence.

The challenge here is that Mnesia, by default, considers dead nodes not to be 
part of the
cluster and carries on operations as if they did not exist. This might be a
desirable behaviour in transactions, and developers can use the \texttt{majority}
option to protect mission-critical data. But in an eventually consistent system,
this is less ideal since we want our system to repair itself automatically.
Therefore Hypermnesia considers dead nodes as part of the cluster and buffers
operations for them.

\subsection{Transient failure} \label{subsec:impl transient failure}

When a transient network failure happens, communication between nodes temporarily
stops but is not long enough for the failure detector to act. Transactions
will completely stall during this period. Although dirty operations can carry
on, replicas might end up in different states due to out-of-order message delivery.
For example, in~\cref{fig:mnesia transient failure}, there might be a
network failure between node B to A, resulting in B's \texttt{add a} being delayed.
If messages are delivered as they arrive, then node A and node C will end up in 
an inconsistent state. This is because addition and deletion in a set 
do not commute, and the two purple operations (add and delete) are concurrent, and
they are applied in a different order on node A and node C, resulting in different final
states.

\begin{figure}[htp]
  \centering
  \begin{subfigure}[t]{0.95\columnwidth}
    \centering
    \begin{tikzpicture}[
      timeline/.style={semithick, dotted},
      commentl/.style={align=right},
      commentr/.style={commentl, align=left},]
      \node[] (n1) { Node A};
      \node[right=1cm of n1] (n2) { Node B};
      \node[right=1cm of n2] (n3) { Node C};
      
      \draw[post] ([yshift=-1cm]n2.south) coordinate (n2add) -- ([yshift=-4cm]n2add-|n1) 
        coordinate (n1addn2) node[pos=.6, above, sloped] {add a};
      \draw[post] (n2add) -- ([yshift=-1cm]n2add-|n3) 
        coordinate (n3addn2) node[pos=.3, above, sloped] {\textcolor{ACMPurple}{add a}};
      \draw[post] ([yshift=-1cm]n1.south) coordinate (n1add) -- ([yshift=-2cm]n1add-|n3)
        coordinate (n3addn1) node[pos=.6, above, sloped] {add a};
      \draw[post] ([yshift=-1.3cm]n1add) coordinate (n1rmv) -- ([yshift=-2cm]n1rmv-|n3)
        coordinate (n3rmvn1) node[pos=.6, above, sloped] {\textcolor{ACMPurple}{delete a}};
      
      \draw[timeline] (n1) -- ($ (n1addn2) + (0, -1cm) $);
      \draw[timeline] (n2) -- ($ (n2|-n1addn2) + (0, -1cm) $);
      \draw[timeline] (n3) -- ($ (n3|-n1addn2) + (0,-1cm) $);
      
      \node[left=2mm of n1add, commentl] {add a};
      \node[below left=0.4cm and 2mm of n1add, commentl] {\{a\}};
      \node[left=2mm of n1rmv, commentl] {delete a};
      \node[below left=0.5cm and 2mm of n1rmv, commentl] {\{\}};
      \node[left=2mm of n1addn2, commentl] {\{a\}};
      \node[left=2mm of n2add, commentl] {add a};

      \node[right=2mm of n3addn2, commentr]  {\{a\}};
      \node[right=2mm of n3addn1, commentr]  {\{a\}};
      \node[right=2mm of n3rmvn1, commentr]  {\{\}};
      \end{tikzpicture}

    \caption{Mnesia replica states can diverge depending on the ordering of delivery of operations.
    The purple operations are causally concurrent.}
    \label{fig:mnesia transient failure}
  \end{subfigure}
  
  \begin{subfigure}[t]{0.95\columnwidth}
    \centering
    \begin{tikzpicture}[
      timeline/.style={semithick,dotted},
      commentl/.style={align=right},
      commentr/.style={commentl, align=left},]
      \node[] (n1) { Node A};
      \node[right=1cm of n1] (n2) { Node B};
      \node[right=1cm of n2] (n3) { Node C};
      
      \draw[post] ([yshift=-1cm]n2.south) coordinate (n2add) -- ([yshift=-4cm]n2add-|n1) 
        coordinate (n1addn2) node[pos=.6, sloped, auto, swap] {(add a, \([0,1,0]\))};
      \draw[post] (n2add) -- ([yshift=-1cm]n2add-|n3) 
        coordinate (n3addn2) node[pos=.6, above, sloped] 
        {\textcolor{ACMPurple}{(add a, \([0,1,0]\))}};
      \draw[post] ([yshift=-1cm]n1.south) coordinate (n1add) -- ([yshift=-2cm]n1add-|n3)
        coordinate (n3addn1) node[pos=.6, above, sloped,near end] {(add a, \([1,0,0]\))};
      \draw[post] ([yshift=-1.3cm]n1add) coordinate (n1rmv) -- ([yshift=-2cm]n1rmv-|n3)
        coordinate (n3rmvn1) node[pos=.7, above, sloped] 
        {\textcolor{ACMPurple}{(delete a, \([2,0,0]\))}};
      
      \draw[timeline] (n1) -- ($ (n1addn2) + (0, -1cm) $);
      \draw[timeline] (n2) -- ($ (n2|-n1addn2) + (0, -1cm) $);
      \draw[timeline] (n3) -- ($ (n3|-n1addn2) + (0,-1cm) $);
      
      \node[left=2mm of n1add, commentl] {add(a)};
      \node[below left=0.4cm and 2mm of n1add, commentl] {\{(a,\([1,0,0]\))\}};
      \node[left=2mm of n1rmv, commentl] {delete(a)};
      \node[below left=0.5cm and 2mm of n1rmv, commentl] {\{\}};
      \node[left=2mm of n1addn2, commentl] {\{(a,\([0,1,0]\))\}};
      \node[left=2mm of n2add, commentl] {add(a)};

      \node[right=2mm of n3addn2, commentr]  {\{(a,\([0,1,0]\))\}};
      \node[right=2mm of n3addn1, commentr]  {\{(a,\([0,1,0]\)), (a,\([1,0,0]\))\}};
      \node[right=2mm of n3rmvn1, commentr]  {\{(a,\([0,1,0]\))\}};
      \end{tikzpicture}
    \caption{Hypermnesia attaches vector timestamps to each message and stores them 
    along with actual elements in the set for conflict resolution.}
    \label{fig:hypermnesia transient failure}
  \end{subfigure}
  \caption{Mnesia and Hypermnesia's response to transient network failure.}
  \label{fig:transient failure}
\end{figure}


In order to achieve convergence, and hence eventual consistency, concurrent
operations need to commute.
The exact semantics of whether addition or deletion wins depends on the actual
application, and to achieve convergence, it is sufficient to
define consistent semantics across replicas. Add-wins semantics is presented here
but the remove-wins semantics is similar. To achieve
add-wins semantics with a pure op-based Set \acrshort{crdt}, we require deletions
to only remove elements that causally precede it, as captured by~\cref{line:awset remove red}
in~\cref{algo:pawset}. In~\cref{fig:hypermnesia transient failure},
the deletion with timestamp \([2,0,0]\) removes the element (a,\([1,0,0]\))
which is causally lower, but not (a,\([0,1,0]\)) which is causally concurrent.

