%TC:ignore
An algorithm for the pure op-based \acrshort{rwset} is shown in~\cref{algo:prwset}.


\begin{algorithm}[htp]
  \DontPrintSemicolon
  \SetKwProg{Fn}{function}{:}{}  

  \SetKwFunction{Add}{add} \SetKwFunction{Delete}{delete} \SetKwFunction{Read}{read}
  \SetKwFunction{Stablise}{stablise}
  \SetKwFunction{Red}{redundant} \SetKwFunction{Reify}{reify}
  \SetKwFunction{RemoveRed}{remove\_redundant}
  \SetKwFunction{App}{append} \SetKwFunction{Rmv}{remove}


  \SetKwData{Msg}{msg} \SetKwData{Sender}{sender}
  \SetKwData{polog}{POLog}
  \SetKwData{true}{true} \SetKwData{false}{false}
  \SetKwData{noop}{no-op} \SetKwData{add}{add} \SetKwData{delete}{delete}
  \SetKwData{clear}{clear}
  \SetKwData{NoDel}{nodel}

  \(\polog \leftarrow []\) \;


  \Fn(){\Add{\(e, t\)}} {
    \RemoveRed{\(e, t, \add\)} \;
    \For{\((e',t',o')\in \polog\)}{
      \If{\Red{\((e,t,\add),(e',t',o')\)}}{
        \KwRet{} \;
      }
    }
    \(\polog \leftarrow \App{\polog, \((e,t,\add)\)} \) \;
  }

  \Fn(){\Delete(\(e, t\))} {
    \RemoveRed{\(e, t, \delete\)}\;

    \For{\((e',t',o')\in \polog\)}{
      \If{\Red{\((e,t,\delete),(e',t',o')\)}}{
        \KwRet{} \;
      }
    }
    \(\polog \leftarrow \App{\polog, \((e,t,\delete)\)} \) \;

  }
   
  \Fn(){\Read{\(k\)}} {
    \For{\((e,t,o) \in \polog\)}{
      \If{\(e.\mathrm{key} = k\)} {
        \(\NoDel \leftarrow \true\) \;
        \For{\(e,t,o \in \polog\)} {
          \If{\(e.\mathrm{key} = k\land o=\delete\)}  {
            \(\NoDel \leftarrow \false\) \;
          }
        }
        \If{\(\NoDel\)} {
          \KwRet{\(e\)} \;
        }
      }
    }
    \KwRet{\(\mathrm{undefined}\)} \;
  }

  \Fn(){\RemoveRed(\(e,t,o\))} {
    \For{\((e',t',o') \in \polog\)}{
      \If{\Red{\((e',t',o'),(e,t,o)\)}}{
        \(\polog \leftarrow \Rmv{\(\polog, (e',t',o')\)}\) \;
      } 
    }
  }

  \Fn(){\Red((e,t,o), (e',t',o'))} {
    \tcc{check whether \((e,t,o)\) is made redundant by \((e',t',o')\)}
    \If{\(e = e' \land t < t'\)}{
        \KwRet{\true} \;
     } \Else{
        \KwRet{\false} \;
     }
  }


  \Fn(){\Reify{\(e,t,o\)}} {
    \RemoveRed{\(e,t,o\)} \;
  }
  \caption{Pure \acrshort{rwset} pseudocode, defined in terms of usual set operations.
  Note this is pseudocode sacrifices efficiency for clarity. Ideas are taken 
  from~\cite{baquero2017PureOp,baquero2014PureOp,bauwens2021Reactivity}.}
  \label{algo:prwset}
\end{algorithm}
%TC:endignore