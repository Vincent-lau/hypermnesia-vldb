\section{Background} \label{sec:bg}

Hypermnesia builds on top of theoretical foundations of consistency models
(\cref{sec:bg consistency models})
and \acrshortpl{crdt} theories (\cref{sec:bg crdt}. It also needs to adapt these frameworks
to suit the existing architecture of Mnesia (\cref{sec:bg mnesia}) 
so as to achieve minimum code refactoring and good engineering practice.


\subsection{Consistency models} \label{sec:bg consistency models}

A distributed system often involves a set of replicated state 
machines~\cite{lamport1978clock}. Coordinating these clusters of machines is a 
challenging task, and tradeoffs can be
made between consistency and performance. Many consistency models are formalised for 
both transactional and non-transactional systems~\cite{viotti2016consistency}. 
This section gives a brief
introduction to three of them in detail: distributed transactions (provided by
Mnesia's transaction API), weak consistency (provided by Mnesia's dirty operations)
and eventual consistency (our aim).

\paragraph{Distributed transactions} is sometimes also called distributed atomic 
commits. It is achieved if all nodes commit
or all nodes abort~\cite{saltzer2009Principles}. This is part of the properties provided by 
ACID~\cite{haerder1983dbtransaction} and
is often implemented with the distributed two-phase commit protocol~\cite{bernstein1987concurrency}.
Mnesia provides such a guarantee via its transaction APIs.

\paragraph{Weak consistency}
is defined as follows:  
``The system does not guarantee that subsequent accesses will return the updated 
value, and several conditions need to be met before the value will be 
returned''~\cite{vogels2008ec,bermbach2013consistency,viotti2016consistency}.
It is (deliberately) vaguely defined to incorporate systems whose replicas
``might become consistent by chance''~\cite{bermbach2013consistency}. 
Mnesia's dirty operations fall into this category. 

\paragraph{Eventual consistency} 
is defined as follows: ``If no new updates are made to the object,
eventually all accesses will return the last updated value''~\cite{vogels2008ec}. 
There are often two steps in achieving eventual consistency~\cite{wikipediacontributors2023ec}:
\begin{enumerate*}[(i)]
  \item anti-entropy~\cite{demers1987epidemic} disseminates data to all the nodes 
  in a cluster;
  \item conflict resolution addresses potential conflicts when handling received data.
\end{enumerate*}

Anti-entropy typically involves some form of broadcasting messages, while conflict
resolution tends to use (one of) the following three techniques:~\cite{kleppmann2017DDIA}:
\emph{\acrfullpl{crdt}}~\cite{shapiro2011CRDT,preguica2018CRDT}
\emph{Mergable persistent data structures}~\cite{farinier2015mergable} and
\emph{Operational transformation}~\cite{ellis1989ot}. 

Among these three ideas, \acrshortpl{crdt} is one of the 
database community's most used and well studied approaches. It has been
successfully deployed in many NoSQL databases, such as 
Riak~\cite{klophaus2010Riak}, Redis~\cite{redis2022rediscrdt} and Microsoft Azure 
Cosmos DB~\cite{shukla2018CosmosDB}. Operational transformation requires a central 
coordinator, which
is unsuitable for Mnesia as it uses a leaderless replication strategy (\cref{subsec:bg mnesia arch}). 
Mergable persistent data structures are based on version
control ideas, but multi-version support is not available in Mnesia,
meaning this method is not easily implementable in Mnesia. In conclusion,
\acrshortpl{crdt} is chosen as the basis of conflict resolution and discussed in
more detail in~\cref{sec:bg crdt}.


\subsection{Mnesia} \label{sec:bg mnesia}

Mnesia is an embedded~\cite{raasveldt2019DuckDB},
specialised distributed \acrshort{dbms} for Erlang/OTP 
applications (\cref{subsec:bg mnesia design goal,subsec:bg mnesia arch}).
It is similar to relational databases where each table stores 
tuples (\cref{subsec:bg mnesia data repr}), but also
different since it uses Erlang as its query language instead of SQL\@.
It has dual mode support for accessing data (\cref{subsec:mnesia access contexts}),
discussed in more detail below.

\subsection{Design goals} \label{subsec:bg mnesia design goal}

The original design of Mnesia attempts to meet several requirements~\cite{ericssonab2023mnesiaguide}:

\begin{enumerate}
  \item Fast real-time key-value lookup
  \item Complex non-real-time queries (mainly for operation and maintenance tasks)
  \item Distributed data (due to the distributed nature of the applications)
  \item High fault tolerance
  \item Dynamic reconfiguration
\end{enumerate}

Based on these, Mnesia was built as part of the Erlang/OTP distribution.
Erlang was originally designed for building fault-tolerant telecommunication systems,
and Mnesia helps it to better achieve that goal by acting as an embedded database: 
it is tightly coupled to Erlang, giving it two distinct features:
\begin{enumerate*}[(a)]
  \item The database runs in the same address space as the application itself,
  offering minimum overhead while accessing data.
  \item The database uses native Erlang records to store its data,
  removing the impedance mismatch between different data formats.
\end{enumerate*}

These special features make Mnesia only suitable for specific purposes.
Mnesia is typically used when there is a need to replicate a 
relatively small amount of data: compared with standard SQL databases that can
handle terabytes of data, Mnesia is instead built for (tens of) gigabytes of 
data~\cite{hebert2013LYSE}.
For example, user login details are often stored as session data, and to 
scale the application out, these data need to be replicated across
nodes and accessed quickly to provide a good user experience. 
Mnesia can be a suitable candidate in such a case due to its compelling performance.

\subsection{Architecture} \label{subsec:bg mnesia arch}

Mnesia is built on top of Erlang's built-in memory and disk term storage
\texttt{ets} and \texttt{dets}~\cite{ericssonab2023stdlib}.  These term storage
can be thought of as primitive storage engines that provide constant (or
logarithmic) access time for large amounts of data~\cite{hebert2013LYSE}.
They support different data structures for storing data, such as set, 
bag, etc.  Internally, these are implemented as hash tables or balanced binary trees.
Mnesia also provides additional functionalities such as transactions and 
distribution on top of \texttt{ets} and \texttt{dets}.

A Mnesia cluster generally has a leaderless architecture where every replica can 
handle client requests. A cluster of Mnesia nodes are connected via the
Erlang distribution protocol, which uses TCP/IP as its carrier by default, providing
reliable in-order delivery. Moreover, the connection is transitive, 
which means they form a cluster of fully connected nodes (or a mesh).


% \begin{figure}[htp]
%   \begin{tikzpicture}[
%     replica/.style={rectangle,rounded corners, draw=black, inner sep=5pt,
%     minimum height=10, minimum width=12, align=center,fill=ACMOrange!50,
%     font={\footnotesize} },
%     client/.style={rectangle,rounded corners,inner sep=5pt, minimum size=15, align=center,
%     draw=black,fill=ACMBlue!50},
%     node distance=2cm
%     ]

%     \node[replica] (n1) {N1};
%     \node[replica, below left=of n1] (n2) {N2};
%     \node[replica, below right=of n1]  (n3) {N3};
%     \node[replica, below right=2cm and 0cm of n2]  (n4) {N4};
%     \node[replica, below left=2cm and 0cm of n3]  (n5) {N5};

%     \path[bidir] (n1) edge (n2);
%     \path[bidir] (n1) edge (n3);
%     \path[bidir] (n1) edge (n4);
%     \path[bidir] (n1) edge (n5);
%     \path[bidir] (n2) edge (n3);
%     \path[bidir] (n2) edge (n4);
%     \path[bidir] (n2) edge (n5);
%     \path[bidir] (n3) edge (n4);
%     \path[bidir] (n3) edge (n5);
%     \path[bidir] (n4) edge (n5);

%   \end{tikzpicture}
%   \caption{Example Mnesia cluster of five nodes. Note they always form a
%   fully connected network.}
%   \label{fig:example mnesia cluster}
% \end{figure}


\begin{figure*}[htp]
    \centering
    \begin{tikzpicture}[
      basemod/.style={rectangle,rounded corners, draw=black, fill=white,inner sep=5pt,
      minimum width=60, minimum height=20, align=center},
      mod/.style={basemod, fill=white},
      newmod/.style={basemod, fill=ACMGreen},
      % every node/.append style = {draw, anchor = west},
      % level 1/.style={sibling distance=3cm},
      level 2/.style={sibling distance=1cm},
      grow via three points={one child at (0.5,-1) and two children 
      at (0.5,-1.5) and (0.5,-2.5)},
      edge from parent path={(\tikzparentnode\tikzparentanchor) |- (\tikzchildnode\tikzchildanchor)},
      edge from parent/.style={draw,semithick},
      newmodedge/.style={ACMGreen}
      ]

      \node[mod, parent anchor=south] {mnesia sup} 
        [anchor=west]
        child {node[mod] {event}}
        child {node[mod] {kernel sup} 
            [grow=right] 
            child[parent anchor=east, child anchor=east] {node {}
              [grow=right,child anchor=west]
              child[parent anchor=west] {node[newmod] {causal} edge from parent[newmodedge]} 
              child[parent anchor=west] {node[newmod] {ec} 
                child[parent anchor=east, child anchor=east] { node{}
                  [grow=right,child anchor=west,level distance=1cm]
                  child[parent anchor=west] {node[newmod] {pawset} edge from parent[newmodedge,dashed]}
                  child[parent anchor=west] {node[newmod] {prwset} edge from parent[newmodedge,dashed]}
                }
                edge from parent[newmodedge]
              } 
              child[parent anchor=west] {node[mod,double copy shadow={opacity=.5}] {others\dots}}
              child[parent anchor=west] {node[mod] {monitor}}  
              child[parent anchor=west] {node[mod] {subscriber}}
              child[parent anchor=west] {node[mod] {locker}}
              child[parent anchor=west] {node[mod, fill=ACMLightBlue] {tm}}
              child[parent anchor=west] {node[mod] {controller}}
            }
        }
        child {node[mod] {ext sup}};
    \end{tikzpicture}

    \caption{Relationships between various Mnesia processes, in the 
    form of a supervision
    tree. Most of the logic for replication is inside \texttt{mnesia\_tm} 
    (highlighted in blue). Modules added by Hypermnesia are highlighted in 
    green. More details on the new modules are
    discussed in~\cref{sec:impl overview}. Diagram partly based 
    on~\cite{mattsson1999mnesiainternal}.}
    \label{fig:mnesia struct}
\end{figure*}



\Cref{fig:mnesia struct} shows a brief overview of the modules in Mnesia according
to the supervision relation.
Erlang's fault tolerance model has the concept of a supervision tree~\cite{ericssonab2023otpdesign},
a hierarchy of processes in charge of monitoring/supervising child processes
for killing and restarting ``misbehaving'' processes. 

\subsection{Data representation} \label{subsec:bg mnesia data repr}

Mnesia stores data in Erlang records~\cite{ericssonab2023stdlib}.
\Cref{tab:erl record} is an example of a student record, and this is how each row 
of a Mnesia table is represented, with~\cref{lst:student rec} as the corresponding definition
in Erlang.

\begin{table}[htp]
\centering
\begin{tabular}{ccccc}
  \toprule
  student & id (key) & name & college & age \\ 
  \midrule
  & bb123 & Bruce Banner & Avengers & 54 \\
  & ts233 & Tony Stark & Avengers & 50 \\
  & sg333 & Steve Rogers & Avengers & 100 \\
  \bottomrule
\end{tabular}
\caption{An example Mnesia table for student information.}
\label{tab:erl record}
\end{table}

\begin{listing}[htp]
\centering
\begin{minted}[frame=lines]{erlang}
  -record(student, 
    {id :: integer(), 
    name:: string(), 
    college :: string(),
    age:: integer()}).
\end{minted}
\caption{Erlang record definition for a student.} 
\label{lst:student rec}
\end{listing}


\subsection{Access contexts} \label{subsec:mnesia access contexts}

A central API provided by Mnesia for table manipulation is given in~\cref{lst:mnesia activity}.
A user calls the \mintinline{erlang}{activity} function which takes in: 

\begin{enumerate}
  \item A \mintinline{erlang}{Kind} of activity, also called a \emph{access context}.
  Currently supported contexts include transactions and dirty operations;
  \item A function (\mintinline{erlang}{Fun});
  \item Arguments to be passed to the function (\mintinline{erlang}{Args});
  \item The module in which the function is defined (\mintinline{erlang}{Mod}).
\end{enumerate}

There are two almost equivalent\footnote{Subtle differences between these two \acrshortpl{api}
can be found in the reference guide~\cite{ericssonab2023mnesiaref}} ways of 
using this API, and \cref{lst:dirty trans api} provides an example of
writing a tuple \verb|{k,v}| into the table \verb|tab_name| and then reading it out.
I use the convention of \verb|function_name/arity| to represent Erlang functions
in these code examples and the rest of this report.


\begin{listing}[htp]
  \centering
  \begin{minted}[frame=lines]{erlang}
  mnesia:activity(Kind, Fun, Args :: [Arg :: term()], Mod) -> t_result(Res) | Res 
    where Kind = activity()
    Fun = fun((...) -> Res)
    Mod = atom()
  \end{minted}
  \caption{Mnesia table manipulation API.}
  \label{lst:mnesia activity}
\end{listing}

\begin{listing}[htp]
  \begin{sublisting}{0.46\linewidth}
    \begin{minted}[frame=lines]{erlang}
      mnesia:activity(transaction, fun () -> 
        mnesia:write({tab_name, k, v}),
        mnesia:read({tab_name, k})
      end).
    \end{minted}
    \caption{transaction with \texttt{activity/2}}
  \end{sublisting}
  \hfill
  \begin{sublisting}{0.46\linewidth}
    \begin{minted}[frame=lines]{erlang}
      mnesia:activity(async_dirty, fun () -> 
        mnesia:write({tab_name, k, v}),
        mnesia:read({tab_name, k})
      end).
    \end{minted}
    \caption{asynchronous dirty with \texttt{activity/2}}
  \end{sublisting}
  \begin{sublisting}{0.46\linewidth}
    \begin{minted}[frame=lines]{erlang}
      mnesia:transaction(fun () -> 
        mnesia:write({tab_name, k, v}),
        mnesia:read({tab_name, k})
      end).
    \end{minted}
    \caption{transaction with \texttt{transaction/1}}
  \end{sublisting}
  \hfill
  \begin{sublisting}{0.46\linewidth}
      \begin{minted}[frame=lines]{erlang}
        mnesia:async_dirty(fun () -> 
          mnesia:write({tab_name, k, v}),
          mnesia:read({tab_name, k})
        end).
      \end{minted}
    \caption{asynchronous dirty with \texttt{async\_dirty/1}}
    \label{sublst:async dirty write}
  \end{sublisting}
  \caption{Comparing transaction and asynchronous dirty operations API.}
  \label{lst:dirty trans api}
\end{listing}


\paragraph{Transactions} in Mnesia provide ACID properties~\cite{ericssonab2023mnesiaguide}.
While this is attractive, it also incurs large performance overhead 
and hence a full-fledged transaction might not be suitable for all tasks.
For example, in a datagram routing application, it can be too slow to initiate
a transaction each time a packet is received~\cite{ericssonab2023mnesiaguide}.
The underlying implementation of Mnesia uses two-phase commit
for distributed transactional atomicity, write-ahead logging for durability and consistency, 
and two-phase locking for isolation.


\paragraph{Dirty operations} in Mnesia bypass most transactional processing
and operate directly on the data. Therefore they are much faster 
(at least \(10\)x~\cite{ericssonab2023mnesiaref}) than transactions. As a consequence,
they lose the ACID properties. However, dirty operations still provide 
some level of consistency such as no garbled records for individual dirty reads.
The underlying implementation of dirty operations follows this pattern:
when the function is called, the operation is first performed
on the local table. Then relevant information is collected including
the Erlang record, the target table, and so on. These are then sent 
to other replicas in the cluster.
Mnesia does not examine the content in the table during this process. This 
property is useful in helping us choose a \acrshort{crdt},
as we will see later that some \acrshortpl{crdt} require examinations of the state
before broadcasting the message (\cref{subsec:impl choose crdt}).
On the receiving side, the transaction manager monitors and
applies the received message accordingly.

\subsection{Consistency and availability} \label{subsec:bg cap}

The CAP theorem~\cite{gilbert2012cap} forces a system to choose between consistency
and availability during a network partition. Mnesia responds to this problem by 
taking \emph{no} stance with respect to the CAP theorem~\cite{andersen2014mnesiacap},
but leaves the handling of the recovery process entirely 
to the developer~\cite{ericssonab2023mnesiaguide}, often resulting in a manual restart
of the Mnesia nodes after partitions. Recent improvements have incorporated
the feature to allow developers to set the \texttt{majority} option which disallows 
any non-dirty operation to a table not in a majority quorum. This option
is useful for preventing conflicts for critical data, but sacrifices the availability
of the database further and only works for transactions. There is also a
\verb|set_master_nodes/2| function which unconditionally loads data from the master node
after a partition. It can be useful in some cases, but is a rather ``brutal''
way of handling partitions since it might lead to data loss.

Erlang's built-in distribution protocol has a failure detector that periodically 
sends heartbeats. If a node does not receive a heartbeat from
another node for some time, then the failure detector considers that node dead
and disconnects itself. Mnesia relies on Erlang's failure detector to function,
and two possible situations can occur during a network partition (NP):

\begin{enumerate}[label={NP\arabic*.},ref={NP\arabic*}]
  \item The partition is a \emph{transient failure} in which the failure 
  detector does not consider
  a temporarily unresponsive node dead. In this case, operations can carry on
  as usual, but transactions will stall since the two-phase commit protocol requires
  responses from all participating nodes. Dirty operations can be performed as usual.
  When the partition heals, buffered messages will then be delivered.
  But dirty operations risk the database being in an inconsistent 
  state when the network heals, as we shall see in~\cref{subsec:impl communication failure}.
  \label{itm:transient partition}
  \item The partition lasts long enough that the failure detector detects the
  \emph{communication failure} and disconnects the nodes it thinks have failed. 
  In this case, Mnesia would 
  reconfigure itself and perform future operations without the failed nodes, i.e.\ Mnesia considers
  itself as having a new cluster with only the members still alive. 
  When the partition heals, Mnesia
  logs an error message saying that the database might be inconsistent
  and ask the developer to fix this issue manually. \label{itm:comm failure}
\end{enumerate}


\subsection{CRDTs} \label{sec:bg crdt}

\acrfullpl{crdt}~\cite{preguica2018CRDT,shapiro2011CRDT} are a family of replicated 
data types with a common set of properties 
that enable operations to be performed locally on each node while always converging
to a final state among replicas if they receive the same set of updates. 
There are two types of \acrshortpl{crdt}: state 
based (\cref{subsec:bg state-based crdt}) and operation 
based (\cref{subsec:bg op-based crdts}). 
Each of them is discussed in detail with examples below.

\subsection{State-based CRDTs}  \label{subsec:bg state-based crdt}

% Here I provide a summary of the properties of state-based \acrshortpl{crdt}. 
% Refer to~\cref{appendix:state-based} 
% and
% for mathematical details and examples. 

A state-based 
\acrshort{crdt}~\cite{vanderlinde2016delta-CRDTs,almeida2018DeltaCRDT,shapiro2011CRDT,preguica2018CRDT} communicates by propagating its state of 
the data structure with other parties, and performs the merge operation with a merge
(sometimes called join) operator to converge towards the \acrfull{lub} of
two states. The merge operator must be commutative, idempotent and
associative to guarantee the convergence property (\cref{prop:state crdt convergence}),
which is key to eventual consistency.

% \begin{proposition}[\cite{shapiro2011CRDT}] \label{prop:state crdt convergence}
%   Any two object replicas of a state-based \acrshort{crdt} eventually converge, 
%   assuming the system transmits payload infinitely often between pairs of replicas over
%   eventually-reliable point-to-point channels.
% \end{proposition}

Propagating the entire states of a (big) data structure can often be expensive.
Therefore \acrfullpl{dcrdt}~\cite{almeida2018DeltaCRDT} is proposed. It
disseminates only the changing parts of the state as 
a \(\delta\)-state, reducing communication costs.
Due to their simple requirements on the channel, \acrshortpl{dcrdt} often
have complex logic in managing and disseminating states.


\subsection{Operation-based CRDTs} \label{subsec:bg op-based crdts}

Operation-based (op-based) \acrshortpl{crdt} send the operation performed on 
the \acrshort{crdt} object (as opposed to the state) along with some metadata.
An op-based \acrshort{crdt} typically requires two \emph{concurrent}
operations \(f\) and \(g\) to 
be commutative under the causal delivery order \(<_d\): 
\(f \parallel_d g \iff f\not <_d g \land g\not <_d f\). 

For this reason, a causal delivery channel is often required for op-based
\acrshortpl{crdt} to work. Given such a channel,  the following 
property (\cref{prop:op crdt convergence}) holds:

\begin{proposition}[\cite{shapiro2011CRDT}] \label{prop:op crdt convergence}
  Any two replicas of an op-based \acrshort{crdt} eventually converge under reliable broadcast
  channels that deliver operations in delivery order \(<_d\).
\end{proposition}

One advantage of op-based \acrshortpl{crdt} is their communication efficiency.
Using a set as an example data structure, instead of sending the
entire set with all its elements,
op-based set can just send operations like \texttt{add} or \texttt{delete}.
This could significantly reduce the communication cost, especially when the set is large.

Given the stronger assumption on the channel, designers of op-based \acrshortpl{crdt}
only need to choose how they want to
order concurrent operations, an example of a particular type (which is also
the one implemented in Hypermnesia) of op-based \acrshortpl{crdt} is given below.

\subsubsection{Pure op-based CRDTs}

The definition of the original op-based \acrshortpl{crdt} makes the separation 
between state-based and op-based \acrshortpl{crdt} less clear in that an op-based
\acrshortpl{crdt} can often include state information while sending operations. 
For example, before sending operation \texttt{add(1)} on a set, the state of the 
set can be inspected and the entire set can be added as the ``operation''. It also has 
a relatively high implementation complexity~\cite{bauwens2021Reactivity}. 
To address this, pure op-based \acrshortpl{crdt} were 
developed~\cite{baquero2017PureOp,baquero2014PureOp}, which forbids inspecting
the state \(\sigma\) while generating the message \(m\).

I now illustrate pure op-based \acrshortpl{crdt} with an example \acrfull{awset}.
\Cref{lst:pure op-based awset} is the conceptual view of implementing it.
Typically there are three operations on an op-based \acrshort{crdt}: a generator
\verb|prepare|, which turns the operation \(o_i\) on the current state \(\sigma_i\) into a 
message \(m_i\) that is sent via reliable causal broadcast; an effector 
\verb|effect|, which receives the message \(m\) and applies it onto the current state \(\sigma_j\)
and produces a new state \(\sigma'_j\); and \verb|eval|, which takes an operation,
such as lookup an element in a set and the current state \(\sigma_i\), and returns
the result of the operation.
For pure op-based \acrshortpl{crdt}, data and timestamps need to be stored in 
a \emph{\acrfull{polog}}~\cite{baquero2017PureOp,baquero2014PureOp},
partially ordered by the timestamp attached to each operation. This aims to 
identify causal relationships between different elements, which is important
for the convergence property (\cref{prop:op crdt convergence}).

\begin{listing}[htp]
  \begin{align*}
    \Sigma: T&\hookrightarrow O \\
    \mathrm{prepare}_i(o, s) &= o \\
    \mathrm{effect}_i(o, t, s) &= s \cup \{(t, o)\} \\
    \mathrm{eval}_i(\mathrm{elems}, s) &= \{v \mid (t, [\mathrm{add}, v]\in s \land 
    \not\exists (t', [\mathrm{rmv}, v]\in s\cdot t < t'))\}
  \end{align*}
  \caption{A pure op-based \acrshort{awset} implementation, adapted 
  from~\cite{baquero2017PureOp}.}
  \label{lst:pure op-based awset}
\end{listing}


The \acrshort{polog} (\(\Sigma\)) for the \acrshort{awset} is a partial function from 
timestamps to operations (with appropriate payload).
Note that the effector takes three arguments,
the operation \(o\), timestamp \(t\) and state \(s\), and it simply unions
the operation and timestamp with the current state (i.e.\ appending it to the log), 
leaving the complexity
to the lookup function. This is a fairly space-expensive operation, since all operations
need to be stored in the log including deletions. Optimisations are discussed
later in \cref{sec:impl pawset}.

