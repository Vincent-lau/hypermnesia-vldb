\section{System design} \label{sec:design}

Hypermnesia is designed to be a drop-in replacement for Mnesia, with the goal
of achieving minimum refactoring of client codebase as well as the Mnesia codebase.
To achieve this, we consider we aspects of our design: the user-facing 
\acrshort{api} design (\cref{subsec:design api}) and the choice of a 
\acrshort{crdt} which
has large impact on how we extend Mnesia (\cref{subsec:design choose crdt}).

% This chapter provides an account of Hypermnesia's design and implementation. 
% Hypermnesia's API and the selection of a \acrshort{crdt} are discussed
% in~\cref{sec:impl design}, with alternative options also considered.



\subsection{API design} \label{subsec:design api}

Hypermnesia's API is designed with the research question \cref{itm:question ec refactor}
on refactoring in mind. We consider the SOLID principle~\cite{martin2000solid}
with the following design goals (DGs):

\begin{enumerate}[label={DG\arabic*.},ref={DG\arabic*}]
  \item \textbf{Backwards compatibility and code refactoring.} Hypermnesia should be
  backwards compatible with the existing codebase and should run normally without
  modification if developers choose not to use the new feature. Moreover, if they opt
  for the new feature, the API should minimise the amount of refactoring 
  needed for an existing codebase. \label{itm:refactor}
  \item \textbf{Single responsibility.} The new feature should be contained in 
  module(s) isolated from existing Mnesia code, while respecting Mnesia's code structure.
  This makes the implementation self-contained and easier to maintain. \label{itm:single resp}
  \item \textbf{Extensibility.} Hypermnesia should be extensible to new 
  implementations. This is beneficial since there are various possible \acrshortpl{crdt}, 
  e.g.\ operation-based and state-based. They often achieve similar goals but rely 
  on different assumptions. Hypermnesia should be extensible to new \acrshort{crdt} 
  variants for different tradeoffs.
  \label{itm:extensible}
\end{enumerate}


Based on the existing APIs provided by Mnesia (\cref{subsec:mnesia access contexts}), 
one way to extend it with new APIs for eventual consistency would be to add a
new access context called \mintinline{erlang}{async_ec} so that database operations
performed within this context are eventually consistent. Using
the same example, we can change~\cref{lst:dirty trans api} (the original
\acrshort{api} for transactions and dirty operations) to~\cref{lst:mnesia ec activity} (the
new \acrshort{ec} \acrshort{api}).
We argue that this is a fairly natural extension to the existing Mnesia access contexts,
and requires little refactoring of existing code (\verb|dirty| or \verb|transaction|
changed to \verb|ec|). It does not break existing features either if no changes
are made to the existing code, thus fulfilling~\cref{itm:refactor}.

\begin{listing}[htp]
  \begin{sublisting}{0.48\columnwidth}
    \begin{minted}[frame=lines]{erlang}
      mnesia:activity(async_ec, fun () -> 
        mnesia:write({tab_name, k, v}),
        mnesia:read({tab_name, k})
      end).
    \end{minted}
    \caption{\texttt{activity/2} with new access context \texttt{async\_ec}}
  \end{sublisting}
  \hfill
  \begin{sublisting}{0.48\columnwidth}
    \begin{minted}[frame=lines]{erlang}
      mnesia:async_ec( fun () -> 
        mnesia:write({tab_name, k, v}),
        mnesia:read({tab_name, k})
      end).
    \end{minted}
    \caption{\texttt{async\_ec} with new function \texttt{async\_ec/1}}
  \end{sublisting}
  \caption{New \acrfull{ec} API based on existing Mnesia APIs, using the same
  example as~\cref{lst:dirty trans api}.} 
  \label{lst:mnesia ec activity}
\end{listing}

We also wish to allow programmers to declare which type of \acrshortpl{crdt} they
want to use. This can be done while declaring the table type as \texttt{pawset}
at creation time. Mnesia asks users to enter the type of each
table at creation time (\cref{lst:mnesia create table}). The default values are 
\texttt{set}, \texttt{bag}, etc. This can be extended
to support pure \acrshort{awset} (\texttt{pawset}) and \acrshort{rwset} (\texttt{prwset}) 
as well 
(\cref{lst:hypermnesia create table}). Note that having multiple Set \acrshort{crdt}
implementations allow developers to choose the most appropriate one for their
applications, demonstrating the extensibility of Hypermnesia (\cref{itm:extensible}).

\begin{listing}[htp]
  \begin{sublisting}[t]{0.46\linewidth}
    \begin{minted}[frame=lines]{erlang}
      mnesia:create_table(project,
                          [{type, set},
                          ...
                          ]).
                          
    \end{minted}
    \caption{Creating a table of default type \texttt{set}.}
    \label{lst:mnesia create table}
  \end{sublisting}
  \hfill
  \begin{sublisting}[t]{0.46\linewidth}
    \begin{minted}[frame=lines]{erlang}
      mnesia:create_table(project,
                          [{type, pawset},
                          ...
                          ]).
    \end{minted}
    \caption{Creating a table of type \texttt{pawset} (pure \acrlong{awset}).
    A \texttt{prwset} (pure \acrlong{rwset}) can be used as well.}
    \label{lst:hypermnesia create table}
  \end{sublisting}
  \caption{New \acrfull{ec} API based on existing Mnesia APIs.} 
  \label{lst:mnesia ec create table}
\end{listing}

Finally, different \acrshort{crdt} logic is implemented inside separate modules,
aiming to achieve the single responsibility requirement (\cref{itm:single resp}).
The new \verb|mnesia_ec| module is built for handling data replication and
interfacing with the underlying \acrshortpl{crdt}, \verb|mnesia_causal| module for
causal delivery, and \verb|mnesia_pawset|/\verb|mnesia_prwset| module(s) for
data storage and conflict resolution (\cref{fig:mnesia struct}).

\subsection{Choosing a CRDT} \label{subsec:design choose crdt}

We talked about different kinds of \acrshortpl{crdt} in~\cref{sec:bg crdt}.
One type of Set \acrshort{crdt} needs to be chosen for Hypermnesia's implementation. 
We consider two main factors (Fac):

\begin{enumerate} [label={Fac\arabic*.},ref={Fac\arabic*}]
  \item How efficient is this \acrshort{crdt} in terms of space and time?
  \item How easy does it fit into the existing Mnesia codebase
  and how many breaking changes need to be made? \label{itm:code change}
\end{enumerate}

On the one hand, state-based \acrshortpl{crdt} (\cref{subsec:bg state-based crdt}) 
have an important drawback in their communication
overhead~\cite{almeida2018DeltaCRDT}. This might not be acceptable for data types 
with large state such as sets. \acrshortpl{dcrdt} is a more suitable candidate for our 
purpose but even with \acrshortpl{dcrdt}, a fair amount
of data needs to be broadcast, especially as the number of operations increases
(see \cref{appendix:state-based} for details). Moreover, it might be difficult to 
examine the state before broadcasting, since Mnesia does not do this by 
default (\cref{subsec:mnesia access contexts}).
State-based \acrshortpl{crdt} do have the advantage of low demand 
on the channel. Therefore reliable broadcasting in Mnesia would be sufficient.
 

On the other hand, operation-based \acrshortpl{crdt} tend to require the channel
to provide causal broadcast, which has to be implemented in Mnesia from scratch.
Dynamic membership, i.e.\
nodes leaving and joining the cluster, is also tricker with operation-based
\acrshortpl{crdt} as buffering and replaying of operations are needed. Nevertheless,
pure operation-based \acrshortpl{crdt} largely resemble Mnesia's anti-entropy strategy,
with no examination of the current content and immediate synchronisation
for each operation~\cite{mattsson2009impltxt}. 
State-based \acrshortpl{crdt} are less suitable for these requirements~\cite{preguica2018CRDT}. 
Therefore pure op-based \acrshortpl{crdt} better meets the second 
requirement (\cref{itm:code change}).

In terms of efficiency, it is challenging to characterise the performance metrics
of each \acrshort{crdt} without practical benchmarking:
\acrshortpl{dcrdt} uses periodic synchronisation of delta states, while
pure op-based \acrshortpl{crdt} go through an extra causal broadcast layer.
Moreover, \citet{almeida2018DeltaCRDT} and \citet{baquero2017PureOp}
proposed optimisations for these \acrshortpl{crdt}. One could
also argue that these two \acrshortpl{crdt} are fundamentally the same as they
are both doing the necessary work for conflict resolution, but at different
layers of the system.

In conclusion, pure op-based \acrshortpl{crdt} is chosen for its operational
similarity with Mnesia's dirty operations. We leave experimenting of
\acrshortpl{dcrdt} as future work (\cref{sec:concl future}).
\Cref{tab:crdt comparison} highlights the differences between these 
two \acrshortpl{crdt}.

\begin{table}[htp]
  \centering
  \begin{tabular}{p{0.3\columnwidth}p{0.3\columnwidth}p{0.3\columnwidth}}
    \toprule
    &   State-based & Operation-based \\ \midrule
    \textbf{Variant of interest} & Delta-state & Pure op-based \\ \midrule
    \textbf{Understandability} & Complex & Medium \\ \midrule
    \textbf{Efficiency} & Good & Good \\ \midrule
    \textbf{Complexity} & Periodic anti-entropy different from Mnesia's protocol & Causal broadcast not provided by Mnesia \\ \midrule
    \textbf{Advantage} & Little requirement on the channel & Similarity to current Mnesia's anti-entropy strategy \\
    \bottomrule
  \end{tabular}
  \caption{Comparison of state and operation based Set \acrshortpl{crdt}.}
  \label{tab:crdt comparison}

\end{table}
