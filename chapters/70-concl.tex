\section{Conclusion} \label{sec:concl}
% As you might imagine: summarizes the dissertation, and draws any
% conclusions. Depending on the length of your work, and how well you
% write, you may not need a summary here.

% You will generally want to draw some conclusions, and point to
% potential future work.

With Hypermnesia, we presented an extension to the Mnesia \acrshort{dbms}
incorporating eventual consistency. 
It allows the database to continue operating while there is a partition and 
automatically resolves potential conflicts after the partition recovers.
Hypermnesia achieves all of this while maintaining a \(10\)--\(20\)x better performance 
over the transaction API. Moreover, its API is designed to minimise the amount of 
code refactoring and fits well into the existing Mnesia API.
In the future we intend to extend Hypermnesia to allow better interactions between
transactions and eventual consistency, and to add support for disk tables.



