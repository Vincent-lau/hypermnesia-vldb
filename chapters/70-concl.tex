\section{Conclusion} \label{sec:concl}
% As you might imagine: summarizes the dissertation, and draws any
% conclusions. Depending on the length of your work, and how well you
% write, you may not need a summary here.

% You will generally want to draw some conclusions, and point to
% potential future work.

With Hypermnesia, we presented an extension to the Mnesia \acrshort{dbms}
incorporating eventual consistency. 
Qualitatively speaking, Hypermnesia's API is designed to minimise the amount of 
code refactoring (\cref{sec:design}) and fits well into the existing Mnesia 
access contexts, making it easier to be adopted in existing open source codebases
(\cref{sec:eval api}) such as RabbitMQ and 
ejabberd~\cite{vmware2023rabbitmq,processone2023ejabberd}.
Moreover, the new eventual consistency API passes the extended Mnesia regression 
test suite (consisting of \(\approx 5000\) unit tests), including (\(\approx 20\)) 
additional tests that cover network partition, \acrshort{awset} and \acrshort{rwset} 
behaviours and causal broadcast (\cref{sec:eval correctness}). It also allows the database to 
continue operating while there is a partition and automatically resolves potential 
conflicts after the partition recovers (\cref{sec:eval fault tolerance}).

Quantitatively speaking, Hypermnesia accomplishes the above functionalities
while providing about \(10\)--\(20\) times higher throughput and 
lower latency than Mnesia's transactions (\cref{sec:eval benchmarks}). Thanks to Mnesia's
performant dirty operations, Hypermnesia can be designed to guarantee eventual 
consistency without sacrificing much performance, taking advantage of the
performant architecture of dirty operations.


\paragraph{Future work}
  At the moment \acrshort{ec} operations do not interact well with Mnesia's
  transactions and dirty operations, i.e.\ they cannot be used on the same table.
  As mentioned in~\cref{subsec:related mvcc}, it is possible to add support for
  transactions that execute and update queries on a
  consistent snapshot and then resolve conflicts between different snapshots
  with \acrshortpl{crdt}, and there are many protocols developed for this 
  purpose~\cite{preguica2014SwiftCloud,shapiro2018Antidote}.
  Now that Mnesia has built-in support for \acrshortpl{crdt} and \acrlong{ec},
  it could be further enhanced to support lightweight but highly available
  transactions. 
  Furthermore, Hypermnesia currently works for in-memory tables only. Although in-memory 
  caching tends to be the way Mnesia is used~\cite{mattsson1998mnesia},
  Hypermnesia could be extended to support disk tables in the future, or even 
  custom backends. Different data structures might open the door 
  for better space optimisation which is currently less feasible 
  with \texttt{ets} and \texttt{dets}.


