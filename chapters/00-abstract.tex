%TC:ignore
\begin{abstract}
    
Mnesia is a soft real-time embedded \acrlong{dbms} written for 
Erlang, a programming language that powers the infrastructures of various organisations 
like Cisco, Ericsson and the NHS. Due to Mnesia's tight integration with 
Erlang, it is also impactful in open source projects such as RabbitMQ and ejabberd. 

However, the development of Mnesia has remained stagnant for years, resulting in the 
lack of features such as automatic conflict resolution: Mnesia leaves the handling 
of conflicts after network partitions entirely to the developer. 
Moreover, as a distributed database, Mnesia only provides two 
extreme forms of consistency guarantee: transactions and weak consistency. 
Existing solutions to this problem are either external libraries or commercial
standalone products, none of which is integrated into Mnesia natively.
This means Erlang developers often have to introduce new dependencies into 
their codebase or resort to less ideal alternative databases.

To address this issue, we propose a new consistency guarantee for Mnesia: \acrfull{ec}. 
The benefit of this is twofold: first, \acrshort{ec} 
introduces an intermediate consistency guarantee between transactions and weak 
consistency, offering more choices to developers; second, this implementation 
of \acrshort{ec} 
with \acrfullpl{crdt} enables automatic conflict resolution after a network partition.

We have implemented \acrshort{ec} as an extension to Mnesia named \emph{Hypermnesia}
and evaluated its correctness, efficiency and usability. 
Evaluation results show that Hypermnesia's \acrshort{ec} operations can produce
consistent results in the presence of partitions and perform more than
\(10\) times faster than Mnesia's default transactions. Moreover, Hypermnesia's 
API enables minimum code refactoring for adoption in real-world systems. We hope
Hypermnesia can be integrated into Mnesia and used by Erlang developers in the future.

\end{abstract}
%TC:endignore