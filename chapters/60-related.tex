\section{Related work} \label{sec:eval}
% This chapter covers relevant (and typically, recent) research
% which you build upon (or improve upon). There are two complementary
% goals for this chapter:
% \begin{enumerate}
%   \item to show that you know and understand the state of the art; and
%   \item to put your work in context
% \end{enumerate}

% Ideally you can tackle both together by providing a critique of
% related work, and describing what is insufficient (and how you do
% better!)

% The related work chapter should usually come either near the front or
% near the back of the dissertation. The advantage of the former is that
% you get to build the argument for why your work is important before
% presenting your solution(s) in later chapters; the advantage of the
% latter is that don't have to forward reference to your solution too
% much. The correct choice will depend on what you're writing up, and
% your own personal preference.

The field of database research is a dynamic and broad domain that encompasses a
variety of topics. This chapter focuses on the architecture and design of eventually 
consistent databases (\cref{sec:related db}).
Similarly, for \acrshortpl{crdt} (\cref{sec:related crdt}), this chapter surveys 
them in the context of how they impact the design and implementation of Hypermnesia.


\subsection{Databases and eventual consistency} \label{sec:related db}

\subsubsection{Mnesia and eventual consistency}

Mnesia's lack of automatic conflict resolution has been problematic for developers 
using it~\cite{mineiro2008DukesofErl,wiger2010unsplitmail}. 
Unsplit~\cite{wiger2023unsplit}, an external library for Mnesia, allows for user-defined
merge logic but requires developers to define the custom logic.
ForgETS~\cite{vorontsov2018forgETS} is WhatsApp's Mnesia ``drop-in'' replacement
database, aiming to provide Mnesia's missing features such as auto reconciliation and auto
reconnection. ForgETS uses the last-write-wins strategy for conflict resolution.
However, ForgETS is a proprietary database influenced by WhatsApp's operational
experience, and the details of its effectiveness are unknown. Moreover, ForgETS
is a standalone database built on top of \texttt{ets}, while
Hypermnesia aims to extend Mnesia to support automatic resolution natively.


\subsubsection{Key-value stores}

Key-value stores like Redis\footnote{\url{https://redis.com}} serve as a caching 
layer between servers and databases, offering an Active-Active architecture that allows
replicated database instances to distribute over different (possibly distant) locations.
Redis is an advanced system with many more features such as expiring keys,
and uses op-based \acrshortpl{crdt} for conflict resolution between different 
instances~\cite{redis2022rediscrdt}. While Mnesia has a similar use case, it is
primitive and embedded into the Erlang ecosystem. 
Mnesia can provide a faster response time than Redis when used on a smaller scale,
thanks to its shared address space with the application. This research aims
to enhance Mnesia with stronger non-transactional consistency models
found in a general purpose in-memory database (like Redis), reducing
the constraints when developers opt for Mnesia as the choice of their Erlang applications.

% mnesia has a specific use case, 
% smaller use case, small amount of data, caching layer
% fast
% experienced developers work with it lack of documentation
% add ec so that developers will use it

\subsubsection{Multi-version concurrency control} \label{subsec:related mvcc}

SwiftCloud~\cite{preguica2014SwiftCloud} is a fault-tolerant geo-replicated 
transactional system. It provides causally
consistent snapshots for each transaction with object versioning. As a geo-replicated system, 
SwiftCloud allows local operations to be performed on the client without going
through the server but only when they operate on mergeable data 
types (e.g.\ \acrshortpl{crdt}) during a transaction. Antidote database~\cite{shapiro2018Antidote} 
goes one step further, integrating \acrshort{tcc} with stronger consistency models and
allowing developers to choose between them.

\acrfull{mvcc} is a popular technique used in conjunction with
\acrshortpl{crdt} to provide eventual or sometimes even transactional consistency.
Mnesia does not support object versioning by default, so
adding \acrshort{mvcc} requires extensive work. In this project, we focus
on adding eventual consistency and leave the integration of \acrshort{mvcc} and 
\acrshortpl{crdt} as future work (\cref{sec:concl future}).


\subsubsection{Augmenting existing embedded databases}

SQLite\footnote{\url{https://sqlite.org/index.html}} is one of the most widely 
used embedded SQL database engines in the world~\cite{hipp2019sqlite}. 
It does not have built-in support for replication, but there has been work that 
extend it for local-first software~\cite{tomter2021SQLitelocal} with
\acrfullpl{crr}: \acrshortpl{crdt} applied to relational databases~\cite{yu2020CausalLen}.
They use a two-layer architecture: an application layer for handling client requests,
and a \acrshort{crr} layer for conflict resolution and anti-entropy protocols.
However, their work still remains in progress and it is not clear how effective their
approach is in terms of overheads. ElectricSQL\footnote{\url{https://electric-sql.com}} 
is another attempt to augment SQLite with local-first behaviours. 
It uses RichCRDT~\cite{balegas2022richcrdt}, which are \acrshortpl{crdt}
extended with database guarantees, based on ideas from Antidote.

These techniques for enhancing SQLite are similar in spirit to Hypermnesia. However,
SQLite is a rather general purpose single-node database receiving constant developments.
In contrast, Mnesia is integrated with the Erlang ecosystem and designed as a 
distributed database, hence they have different use cases.
Moreover, to the best of our knowledge, there is no performance evaluation on 
these extensions to SQLite yet.

\subsection{CRDTs research} \label{sec:related crdt}

\subsubsection{Systems using CRDTs} \label{subsec:related crdt system}

\acrshortpl{crdt} are often used in collaboration software such
as shared text editors~\cite{weiss2010Logoot} to provide local-first 
behaviours~\cite{kleppmann2019local-first}, i.e.\ users
of the editors can keep typing into the document despite unstable networks.
It has then found its use in areas such as synchronisation on the edge and
opportunistic networks~\cite{guidec2023opportunistic,yu2020crr}. 
These systems typically have constrained network or computing resources and 
\acrshortpl{crdt} can be used to delay synchronisation and respond to the user first. 

Researchers have been investigating new areas where \acrshort{crdt} techniques
can be applied. Hypermnesia does not aim to apply \acrshortpl{crdt} in a novel
way but rather to experiment with their suitability in an embedded database like Mnesia.

\subsubsection{Time and space improvements} \label{subsec:related crdt performance}

\acrshortpl{crdt} are constantly invented and improved to reduce their resource 
consumption.
For example, the proposal of \acrshortpl{dcrdt}~\cite{almeida2018DeltaCRDT} is partly
due to the 
communication overhead of propagating the entire state in normal state-based
\acrshortpl{crdt}. Pure op-based \acrshortpl{crdt}~\cite{baquero2014PureOp} were also 
proposed to reduce the space overhead. \citet{bauwens2021Reactivity}
further improve the reactivity and storage requirements of pure op-based
\acrshortpl{crdt} by exposing more information from the causal broadcast 
layer. \citet{vanderlinde2016delta-CRDTs} suggest a new way to 
adapt \acrshortpl{dcrdt} for more unstable networks.
Furthermore, \citet{bauwens2019crdtmemory} show a more eager way to remove metadata
using acknowledgements from replicas.

One research question of this project is to understand the overhead of
eventual consistency in Mnesia. The space and time optimisations proposed are 
useful for improving Hypermnesia's efficiency,
therefore they are investigated in detail, and some of their ideas are
used in Hypermnesia's implementation (\cref{sec:impl pawset}).

