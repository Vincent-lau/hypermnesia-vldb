\section{Introduction} \label{sec:intro}

% This is the introduction where you should introduce your work. In
% general the thing to aim for here is to describe a little bit of the
% context for your work -- why did you do it (motivation), what was the
% hoped-for outcome (aims) -- as well as trying to give a brief overview
% of what you actually did.

% It's often useful to bring forward some ``highlights'' into this
% chapter (e.g.\ some particularly compelling results, or a particularly
% interesting finding).

% It's also traditional to give an outline of the rest of the document,
% although without care this can appear formulaic and tedious. Your
% call.


Mnesia is an embedded distributed \acrfull{dbms} designed for industrial-grade
communication systems written in Erlang~\cite{ericssonab2023mnesiaguide}. It is
influential in its usage across companies and organisations like Cisco,
Goldman Sachs and Mastercard~\cite{cesarini2019erlang}.
It is also used in open source highly scalable message brokers and XMPP services, such as 
RabbitMQ~\cite{vmware2023rabbitmq}, MongooseIM~\cite{erlangsolutions2023MongooseIM} and 
ejabberd~\cite{processone2023ejabberd}. Furthermore, Mnesia is part of the \acrfull{lyme}
software bundle for building dynamic web pages that can handle heavy 
traffic~\cite{wikipediacontributors2020LYME}.

Despite its usage in heavy-load applications such as WhatsApp~\cite{vorontsov2018forgETS}
that handle tens of thousands of messages per second~\cite{levy2022RabbitMQvsKafka}, the development of Mnesia has stayed
relatively stagnant for many years\footnote{Much of the core code modified in this 
project are over 13 years old, dating back to the time when Erlang was published on Github.}.
Due to Mnesia's highly specialised nature and complex codebase, extending it often requires
a combination of research knowledge and language expertise. This has
deterred many developers from contributing to Mnesia,
which means Mnesia lacks features that many modern distributed databases have. 
Examples include the lack of non-transactional
guarantees and automatic recovery after network partitions.

Mnesia provides two ways to access the database: transactions
and dirty operations. Transactions provide ACID guarantees~\cite{haerder1983dbtransaction}, 
while dirty operations only give weak consistency~\cite{vogels2008ec}. Developers 
have to choose either the
strong transactional API while sacrificing performance and availability during
network partitions, or use dirty operations for fast performance, but risk their
data being inconsistent.  Moreover, Mnesia leaves the handling of potential
data inconsistency after a partition entirely up to the 
developer~\cite{ericssonab2023mnesiaguide}, which means a manual 
restart of the cluster is often needed. Indeed, developers have reported that
\say{\textit{the experience has convinced us that we need to prioritize up our 
network partition recovery strategy}}~\cite{mineiro2008DukesofErl} 
while handling a partitioned Mnesia cluster. 


There are plenty of consistency models that are weaker than transactions but stronger
than dirty operations' weak consistency~\cite{ericssonab2023mnesiaref}. 
One such model is eventual consistency~\cite{vogels2008ec}, which
guarantees convergence eventually when no updates are made to the data. This is an attractive
model for Mnesia since it often has better performance than transactions, while
also maintaining consistency eventually. Modelling data this way can serve
as a viable approach for achieving data consistency after a network partition.

To this end, I propose the following research questions (RQs):

\begin{enumerate}[label={RQ\arabic*.},ref={RQ\arabic*}]
  \item \emph{Can we introduce a non-transactional consistency guarantee, for example,
  eventual consistency, into Mnesia?} Mnesia was developed and designed before many
  of these consistency models became popular~\cite{mattsson1998mnesia}, and therefore 
  retrofitting them into Mnesia can be a non-trivial task.  \label{itm:question ec possible}
  \item \emph{How much overhead, in terms of time and space, will this new guarantee
  bring, compared to Mnesia's existing operations?} Mnesia's tight integration with
  Erlang gives it outstanding performance compared to other standalone databases.
  Therefore, it is important to evaluate the overhead of the new operations to
  ensure they are still competitive. \label{itm:question ec overhead}
  \item \emph{Can eventual consistency help us achieve automatic conflict resolution
  after a partition in Mnesia?} The fault tolerance model of 
  Mnesia (\cref{sec:impl fault tolerance}) may present extra challenges in this 
  otherwise straightforward extension. \label{itm:question ec conflict}
  \item \emph{How much code refactoring is needed for the consistency guarantee to
  function?} Mnesia is used in many large-scale (legacy) codebases, which might
  be too expensive to do refactoring. Therefore it is important to understand 
  the cost of adopting a new consistency model API in Mnesia.  
  \label{itm:question ec refactor}
\end{enumerate}


\subsection{Challenges} \label{sec:intro challenges}

There are several technical challenges (TCs) that need to be overcome to design,
implement and evaluate Hypermnesia:

\begin{enumerate}[label={TC\arabic*.}, ref={TC\arabic*}]
  \item Mnesia is a rather old database system that has not received much attention for
  years. This means the codebase contains legacy code and is not well documented.
  \label{itm:intro mnesia old}
  \item The previous point implies that there are potentially legacy projects
  using Mnesia. The new feature, therefore, needs to be backwards compatible and 
  ideally easy to be adopted by developers.
  \item Another consequence of~\cref{itm:intro mnesia old} is that its test
  suite does not include network partition tests, which are essential in testing
  the correctness of eventual consistency. Moreover, its benchmark suite is only
  written for transactions. Both of these need to be extended to evaluate Hypermnesia.
  \item There are multiple ways to achieve eventual consistency. Therefore an
  informed choice needs to be made based on Mnesia's existing architecture. A survey 
  of existing methods and a study of the
  relatively undocumented architecture of Mnesia is essential.
\end{enumerate}


\subsection{Contributions} \label{sec:intro contributions}

\begin{itemize}
  \item \textbf{A new eventual consistency model for the Mnesia \acrshort{dbms}.}
  A new set of \acrshortpl{api} is designed for this new model to minimise the 
  amount of code refactoring for existing codebases. Moreover, with this new API, 
  automatic conflict resolution is
  possible after a network partition, relieving developers from the burden
  of manually restarting the database.
  \item \textbf{An extension of the test suite for network partitions.}
  These are used to test the correctness of Hypermnesia. It might
  be useful for testing existing functionalities such as transactions as well.
  \item \textbf{An extension of the existing benchmarking library for dirty operations
  and \acrshort{ec} operations.} These are built on top of the original benchmarks 
  for transactions.
  \item \textbf{An evaluation of the new eventual consistency API}, in terms of the
  following:
  \begin{enumerate*}[(a)]
    \item fault tolerance by testing its correctness under network partitions
    and ability to reconcile after a partition;
    \item performance by analysing time and space overhead and comparing against 
    existing operations. Results show that Hypermnesia's \acrshort{ec} operations
    can often achieve \(10\)x higher throughput and lower latency than 
    Mnesia's transactions;
    \item usability by analysing the amount of code refactoring needed to adopt
    the new API in real codebases.
  \end{enumerate*} 

\end{itemize}

